\documentclass[a4paper]{article}
\usepackage{xeCJK}
\usepackage{bm}
\usepackage{amsmath}
\usepackage{amsfonts}
\usepackage{amssymb}
\usepackage{color}
\usepackage[colorlinks=true]{hyperref}
\setCJKmainfont{STKaiti}
\setCJKsansfont{STKaiti}
\setCJKmonofont{Monaco}

\begin{document}
\title{Coursera 機器學習技法}
\maketitle

\section{Week 1}
\subsection{SVM 解决的问题}
动机:提高线性模型对随机噪声的抵抗力,减少 overfit 的可能——选择 margin 最大的分类面。

找最大 margin 即找同时满足两个条件的分类面:
\begin{enumerate}
  \item 正确分类所有的点
  \item 所有数据点中离分类面最近的点到分类面的距离最大
\end{enumerate}

设 $x^{\prime}, x^{\prime\prime}$ 在平面 $\mathbf{w}^{T}\mathbf{x}^{\prime}$ 上,
则
$\mathbf{w}^{T}\mathbf{x}^{\prime} = -b$,
$\mathbf{w}^{T}\mathbf{x}^{\prime\prime} = -b$
故
$\mathbf{w}^{T}\mathbf{x^{\prime} - x^{\prime\prime}} = 0$,即 $\mathbf{w}$ 为法线。
所以求任意一点 $\mathbf{x}$ 到平面距离即求 $\mathbf{x - x^{\prime}}$ 在法线上的投影:

$$d = \frac{1}{\|\mathbf{w}\|}|\mathbf{w}^{T}\mathbf{x} + b|$$

而且要分类正确,所以 $y_n(\mathbf{w}^{T}\mathbf{x}_n + b) > 0$,上式简化为

$$dist = \frac{1}{\|\mathbf{w}\|}y_n(\mathbf{w}^{T}\mathbf{x}_n + b)$$

问题转化为

\begin{equation}
\begin{aligned}
\max_{b, \mathbf{w}} \min_{n=1,\dots,N} \quad &\frac{1}{\|\mathbf{w}\|}y_n(\mathbf{w}^{T}\mathbf{x}_n + b) \\
\text{subject to} \quad& y_n(\mathbf{w}^{T}\mathbf{x}_n + b) > 0 \quad \text{for all } n
\end{aligned}
\end{equation}

注意到,超平面方程 $\mathbf{w}^{T}\mathbf{x} + b = 0$ 两边同乘一个倍数后保持不变,所以
上式中可以通过 scaling 来使得 $\displaystyle \min_{n=1,\dots,N}y_n(\mathbf{w}^{T}\mathbf{x}_n + b) = 1$,
所以问题变为

\begin{equation}
\begin{aligned}
\max_{b, \mathbf{w}} \quad& \frac{1}{\|\mathbf{w}\|} \\
\text{subject to} \quad& \min_{n=1,\dots,N}y_n(\mathbf{w}^{T}\mathbf{x}_n + b) = 1 \\
\end{aligned}
\end{equation}

然后把约束条件放宽为对所有 $n=1,\dots, N$ 满足 $y_n(\mathbf{w}^{T}\mathbf{x}_n + b) \ge 1$。这样做不会改变
原问题的最优解,因为若最优解 $b, \mathbf{w}$ 使得所有 $n=1,\dots,N$ 都有
$y_n(\mathbf{w}^{T}\mathbf{x}_n + b) = t > 1$,则可将 $b, \mathbf{w}$ scale 为原来的
$\displaystyle \frac{1}{t}$ 让它们满足原约束,并得到更优解 $\displaystyle \frac{b}{t}, \frac{\mathbf{w}}{t}$,
这就与 $b, \mathbf{w}$ 是最优解的假设矛盾。所以最优解只可能在原约束下取到。

问题再次转变为

\begin{equation}
\begin{aligned}
\min_{b, \mathbf{w}} \quad& \frac{1}{2}\mathbf{w}^{T}\mathbf{w} \\
\text{subject to} \quad& y_n(\mathbf{w}^{T}\mathbf{x}_n + b) \ge 1 \quad \text{for all } n\\
\end{aligned}
\end{equation}


\subsection{线性 SVM 和 QP}
以上线性 SVM 的最优化问题即 quadratic programming 问题。一般形式为

\begin{equation}
\begin{aligned}
\min_{\mathbf{u} \in \mathbb{R}^{L}} \quad& \frac{1}{2}\mathbf{u}^{T}\mathbf{Qu} + \mathbf{p}^{T}\mathbf{u} \\
\text{subject to} \quad& \mathbf{a}_m^{T}\mathbf{u} \ge c_m \quad (m = 1, \dots, M)\\
\end{aligned}
\end{equation}

当 $\mathbf{Q}$ 正定时 QP 问题为凸。QP 可以直接用凸优化的库来计算。通常需要改写为矩阵形式:

\begin{equation}
\begin{aligned}
\min_{\mathbf{u} \in \mathbb{R}^{L}} \quad& \frac{1}{2}\mathbf{u}^{T}\mathbf{Qu} + \mathbf{p}^{T}\mathbf{u} \\
\text{subject to} \quad& \mathbf{Au} \ge \mathbf{c}
\end{aligned}
\end{equation}

然后扔给 QP solver 算算算……凸优化还是要学习的……不然不知道原理啊心慌慌。

对线性 hard-margin SVM 问题,QP 的这些参数分别为

\begin{equation}
\begin{aligned}
\mathbf{Q} &=
\begin{bmatrix}
0 & \mathbf{0}_d^{T} \\
\mathbf{0}_d & \mathbf{I}_d \\
\end{bmatrix} \\
\mathbf{A} &=
\begin{bmatrix}
y_1 & y_1\mathbf{x}_1^{T} \\
\vdots & \vdots \\
y_N & y_N\mathbf{x}_N^{T} \\
\end{bmatrix} \\
\mathbf{p} &= \mathbf{0}_{d+1} \\
\mathbf{c} &= \mathbf{1}_N
\end{aligned}
\end{equation}

此时留给 QP 的 $\mathbf{Q}$ 矩阵维度为 $d + 1$,与数据维度相关。



\subsection{SVM 的优势}

总的来讲,SVM 相当于在保证 $E_{in} = 0$ 的情况下优化一个惩罚项,
而且较大的 margin 能够有效地限制模型的复杂度,同时令交叉验证时得到的 $E_{CV}$
也会受到限制。

LFD 书上给出了 margin 为 $\rho$ 的 SVM 的 VC 维

$$\displaystyle d_{VC}(\rho) \le \left\lceil \frac{R^2}{\rho^2} \right\rceil + 1$$

其中 $R$ 为距原点最远的点到原点的距离。

证明看 LFD 书和课后练习。大致思路是分别讨论 SVM 能 shatter 的点数 $N$ 的奇偶性,
然后根据每个点到分类面的距离至少为 $\rho$ 得到不等式并求和,右边用 Cauchy-Schwarz 不等式放大,
再用概率证明存在,得到右边的上界即可。

做 leave-one-out 交叉验证时,$\displaystyle E_{CV} = \frac{1}{N}\sum_{n=1}^{N}e_n$,
如果去掉的点不在 margin 边界上,则肯定分类正确,所以 $e_n = 0$,而 margin
边界上的点有 $e_n \le 1$,所以 SVM 的 

$$\displaystyle E_{CV} = \frac{1}{N}\sum_{n=1}^{N}e_n \le \frac{\text{\# support vectors}}{N}$$

这个上界不依赖数据维度,使得 SVM 可以配合特征变换,在高维空间得到
非线性的分类边界,同时限制复杂度,降低普通线性模型加特征变换可能产生的过拟合问题。

特征变换之后的 SVM 问题为

\begin{equation}
\begin{aligned}
\min_{\tilde{b}, \tilde{\mathbf{w}}} \quad& \frac{1}{2}\tilde{\mathbf{w}}^{T}\tilde{\mathbf{w}} \\
\text{subject to} \quad& y_n(\tilde{\mathbf{w}}^{T}\mathbf{z}_n + \tilde{b}) \ge 1 \quad \text{for all } n\\
\end{aligned}
\end{equation}


\subsection{Dual SVM 的动机}
特征变换后的 SVM QP 问题有 $\tilde{d}+1$ 个变量:$\tilde{b}$ 和 $\tilde{\mathbf{w}}$ 和 $N$ 个约束,
依赖于训练数据的维度和变换后空间的维度。当 $\tilde{d}$ 太大时难以计算。
所以,理想的情形是,变换之后的计算不依赖于 $\tilde{d}$。
根据凸优化,可通过解原问题的 Lagrange dual 来获得原问题最优解。

\subsection{QP Lagrange Dual}
假设待解问题为

\begin{equation}
\begin{aligned}
\min_{\mathbf{u} \in \mathbb{R}^{L}} \quad& \frac{1}{2}\mathbf{u}^{T}\mathbf{Qu} + \mathbf{p}^{T}\mathbf{u} \\
\text{subject to} \quad& \mathbf{a}^{T}\mathbf{u} \ge c
\end{aligned}
\end{equation}

相关联的优化问题为

\begin{equation}
\min_{\mathbf{u} \in \mathbb{R}^{L}} \frac{1}{2}\mathbf{u}^{T}\mathbf{Qu} + \mathbf{p}^{T}\mathbf{u} + \max_{\alpha \ge 0} \alpha(c - \mathbf{a}^{T}\mathbf{u})
\end{equation}

LFD exercise 8.9 证明了此问题和原问题的最优解相同,都能满足原问题的约束。
直观来看嘛,上面第二个式子构造出来一个新函数,里面嵌入原问题的约束。最优解肯定要满足约束,
所以后面那个 max 一坨肯定是 0,也就是说,构造出来的新函数让约束得到满足,
同时不改变原函数的取值。

将上式改写为 Lagrange 函数

$$\mathcal{L}(\mathbf{u}, \alpha) = \frac{1}{2}\mathbf{u}^{T}\mathbf{Qu} + \mathbf{p}^{T}\mathbf{u} + \alpha(c - \mathbf{a}^{T}\mathbf{u})$$

则要优化的问题为

$$\min_{\mathbf{u}} \max_{\alpha \ge 0} \mathcal{L}(\mathbf{u}, \alpha)$$

此问题的 strong dual 为

$$\max_{\alpha \ge 0} \min_{\mathbf{u}} \mathcal{L}(\mathbf{u}, \alpha)$$

交换最大最小之后,就可以在没有约束的情况下求解内层,然后再解外层。

多个约束存在时,要将每个约束都用 Lagrange 乘数加入 Lagrange 函数,即所谓的 Karush-Kühn-Tucker(KKT)定理。
求得的最优解满足 KKT 条件:

设待优化问题为

\begin{equation}
\begin{aligned}
\min_{\mathbf{u} \in \mathbb{R}^{L}} \quad& \frac{1}{2}\mathbf{u}^{T}\mathbf{Qu} + \mathbf{p}^{T}\mathbf{u} \\
\text{subject to} \quad& \mathbf{a}_m^{T}\mathbf{u} \ge c_m \quad (m = 1, \dots, M)\\
\end{aligned}
\end{equation}

定义 Lagrange 函数

$$\mathcal{L}(\mathbf{u}, \bm{\alpha}) = \frac{1}{2}\mathbf{u}^{T}\mathbf{Qu} + \mathbf{p}^{T}\mathbf{u} + \sum_{m=1}^{M}\alpha_m(c_m - \mathbf{a}_m^{T}\mathbf{u})$$

则 $\mathbf{u}^{*}$ 是原问题最优解 iff $\mathbf{u}^{*}, \mathbf{\alpha}^{*}$ 是 dual

$$\max_{\bm{\alpha} \ge \bm{0}} \min_{\mathbf{u}} \mathcal{L}(\mathbf{u}, \bm{\alpha})$$

的最优解。而且 dual 的最优解满足
\begin{enumerate}
  \item Primal 和 dual 的约束:$\mathbf{a}_m^{T}\mathbf{u}^{*} \ge c_m$ 和 $\alpha_m \ge 0$。其中 $m=1,\dots,M$
  \item Complementary slackness:$\alpha_m^{*}(\mathbf{a}_m^{T}\mathbf{u}^{*} - c_m) = 0$
  \item Stationarity w.r.t. $\mathbf{u}$:$\nabla_{\mathbf{u}}\mathcal{L}(\mathbf{u}, \bm{\alpha})|_{\mathbf{u}=\mathbf{u}^{*}, \bm{\alpha}=\bm{\alpha}^{*}} = 0$
\end{enumerate}


\subsection{Dual SVM}
根据 SVM 的优化目标和约束构造出对应的 dual 即可。

根据 SVM 的问题

\begin{equation}
\begin{aligned}
\min_{b, \mathbf{w}} \quad& \frac{1}{2}\mathbf{w}^{T}\mathbf{w} \\
\text{subject to} \quad& y_n(\mathbf{w}^{T}\mathbf{x}_n + b) \ge 1 \quad \text{for all } n\\
\end{aligned}
\end{equation}

得到 Lagrange 函数

\begin{equation}
\begin{aligned}
\mathcal{L}(b, \mathbf{w}, \bm{\alpha}) &= \frac{1}{2}\mathbf{w}^{T}\mathbf{w} + \sum_{n=1}^{N}\alpha_n(1 - y_n(\mathbf{w}^{T}\mathbf{x}_n + b)) \\
                                            &= \frac{1}{2}\mathbf{w}^{T}\mathbf{w} - \sum_{n=1}^{N}\alpha_n y_n \mathbf{w}^{T}\mathbf{x}_n - b \sum_{n=1}^{N}\alpha_n y_n + \sum_{n=1}^{N}\alpha_n
\end{aligned}
\end{equation}

先解内层没有约束的最小化问题:
\begin{equation}
\begin{aligned}
\frac{\partial\mathcal{L}}{\partial b} &= -\sum_{n=1}^{N}\alpha_n y_n \\
\frac{\partial\mathcal{L}}{\partial \mathbf{w}} &= \mathbf{w} - \sum_{n=1}^{N}\alpha_n y_n \mathbf{x}_n
\end{aligned}
\end{equation}

得到内层最优需要满足的条件

\begin{equation}
\begin{aligned}
& \sum_{n=1}^{N}\alpha_n y_n = 0\\
& \mathbf{w} = \sum_{n=1}^{N}\alpha_n y_n \mathbf{x}_n
\end{aligned}
\end{equation}

这里得到了一个关于 $\alpha$ 的限制,因为如果 $\alpha$ 不满足此条件,
$-b \sum_{n=1}^{N}\alpha_n y_n$ 可以通过取 $b$ 的值将 $\mathcal{L}$ 变为 $-\infty$。

将内层问题得到的 $\mathbf{w}$ 回带入原问题,得

\begin{equation}
\begin{aligned}
  &\frac{1}{2}\mathbf{w}^{T}\mathbf{w} - \sum_{n=1}^{N}\alpha_n y_n \mathbf{w}^{T}\mathbf{x}_n \\
= &\frac{1}{2} \sum_{n=1}^{N}\alpha_n y_n \mathbf{x}_n^{T} \sum_{m=1}^{N}\alpha_m y_m \mathbf{x}_m - \sum_{n=1}^{N}\alpha_n y_n \sum_{m=1}^{N}\alpha_m y_m \mathbf{x}_m^{T}\mathbf{x}_n \\
= &\frac{1}{2} \sum_{n=1}^{N} \sum_{m=1}^{N} y_n y_m \alpha_n \alpha_m \mathbf{x}_n^{T} \mathbf{x}_m - \sum_{n=1}^{N} \sum_{m=1}^{N} y_n y_m \alpha_n \alpha_m \mathbf{x}_n^{T} \mathbf{x}_m \\
= &-\frac{1}{2} \sum_{n=1}^{N} \sum_{m=1}^{N} y_n y_m \alpha_n \alpha_m \mathbf{x}_n^{T} \mathbf{x}_m
\end{aligned}
\end{equation}

最终 Lagrange 函数变为

$$\mathcal{L}(\mathbf{\alpha}) = -\frac{1}{2} \sum_{n=1}^{N} \sum_{m=1}^{N} y_n y_m \alpha_n \alpha_m \mathbf{x}_n^{T} \mathbf{x}_m + \sum_{n=1}^{N}\alpha_n$$

取个负号,目标问题变为

\begin{equation}
\begin{aligned}
\min_{\mathbf{\alpha} \in \mathbb{R}^{N}} \quad &\frac{1}{2} \sum_{n=1}^{N} \sum_{m=1}^{N} y_n y_m \alpha_n \alpha_m \mathbf{x}_n^{T} \mathbf{x}_m - \sum_{n=1}^{N}\alpha_n \\
\text{subject to} \quad &\sum_{n=1}^{N}y_n\alpha_n = 0\\
                        &\alpha_n \ge 0 (n = 1, \dots, N)\\
\end{aligned}
\end{equation}

现在终于编程了一般的 QP 问题,直接扔给 QP solver 即可,参数为:
\begin{equation}
\begin{aligned}
\mathbf{Q}_D &=
    \begin{bmatrix}
        y_1y_1\mathbf{x}_1^{T}\mathbf{x}_1 & \cdots & y_1y_N\mathbf{x}_1^{T}\mathbf{x}_N \\
        y_2y_1\mathbf{x}_2^{T}\mathbf{x}_1 & \cdots & y_2y_N\mathbf{x}_2^{T}\mathbf{x}_N \\
        \vdots & \vdots & \vdots \\
        y_Ny_1\mathbf{x}_N^{T}\mathbf{x}_1 & \cdots & y_Ny_N\mathbf{x}_N^{T}\mathbf{x}_N \\
    \end{bmatrix} \\
\mathbf{A}_D &=
    \begin{bmatrix}
        \mathbf{y}^{T} \\
        -\mathbf{y}^{T} \\
        \mathbf{I}_{N \times N} \\
    \end{bmatrix} \\
\mathbf{p} &= -\mathbf{1}_{N} \\
\mathbf{c} &= \mathbf{0}_{N+2}
\end{aligned}
\end{equation}

这里的矩阵 $\mathbf{Q}_D$ 一般非零元素很少,所以数据量 $N$ 较大时需要耗费大量内存
来存储中间结果,而且计算也比较吃力。同时,虽然 $\mathbf{Q}_D$ 的维度与 $\tilde{d}$
无关,但其中每个元素要算内积,特征转换之后算内积的维度更大,所以其实还是与 $\tilde{d}$
有关系。

在此问题下求解外层最优 $\mathbf{\alpha}$,然后代入 $\mathbf{w}$ 即可求得最优。

假设训练数据正负例都有(否则不用分类了),那么最优解中至少有一个 $\alpha_s > 0$。
由 KKT 条件中的 complementary slackness 可知,$\alpha$ 和约束在最优解时必有一个为 0,
所以这个 $\alpha_s > 0$ 必须满足

$$y_n(\mathbf{w}^{T}\mathbf{x}_s + b) = 1$$

由此即可解出 $b$

$$b^{*} = y_s - \sum_{\alpha_n^{*} > 0} y_n\alpha_n^{*} \mathbf{x}_n^{T} \mathbf{x}_s$$

这里可知,只有满足 $\alpha_n > 0$ 才会为分类面做贡献。它们对应的数据点即成为支持向量。

将 $\mathbf{w}$ 和 $b$ 代入分类面的方程,即可得到最终输出的 hypothesis 为

$$g(\mathbf{x}) = sign \left ( \sum_{\alpha_n^{*} > 0} y_n \alpha_n^{*} \mathbf{x}_n^{T} \mathbf{x} + b^{*} \right )$$

注意,支持向量只是最终决定分类面的 candidate,分类面 margin 边界上有些点可能对应的 $\alpha = 0$,因此并未对分类面做贡献。
于是之前的交叉验证 error 可以进一步被只满足 $\alpha > 0$ 的支持向量限制。实际中支持向量的数量可能并不多,所以 dual 问题
为非线性的特征转换提供了便利:既能利用特征变换做出复杂的分类边界,又能保证模型复杂度受到限制。


\section{Week 2}
\subsection{Kernel Trick}
Kernel 的用途是将前面 dual SVM 中矩阵 $\mathbf{Q}_D$ 的内积计算和非线性变换加以结合,
不用显式地做特征变换,同时直接计算内积,使计算彻底与 $\tilde{d}$ 无关。一般形式为:

$$K_{\Phi}(\mathbf{x}, \mathbf{x}^{\prime}) \equiv \Phi(\mathbf{x})^T \Phi(\mathbf{x}^{\prime})$$

\subsection{多项式 Kernel}
以 second-order polynomial kernel 为例推导一下:

$$\Phi_2(\mathbf{x})^T \Phi_2(\mathbf{x}^{\prime}) = 1 + \sum_{i=1}^{d}x_ix_i^{\prime} + \sum_{i=1}^{d}\sum_{j=1}^{d}x_ix_jx_i^{\prime}x_j^{\prime}$$

最后一项

$$\sum_{i=1}^{d}\sum_{j=1}^{d}x_ix_jx_i^{\prime}x_j^{\prime} = \sum_{i=1}^{d}x_ix_i^{\prime} \times \sum_{j=1}^{d}x_jx_j^{\prime} = (\mathbf{x}^T\mathbf{x}^{\prime})^2$$

所以

$$K(\mathbf{x}^T, \mathbf{x}^{\prime}) = 1 + (\mathbf{x}^T\mathbf{x}^{\prime}) + (\mathbf{x}^T\mathbf{x}^{\prime})^2$$

可见这里直接在训练数据的空间中算内积,就能直接得出变换后空间的内积,不用做变换了。
前面导出的 SVM 的公式中,直接用 kernel 换掉内积即得带 kernel 简化的 SVM。
除 SVM 之外的其他学习算法需要计算内积的地方都可以从 kernel 获益。

Kernel 做的事情实际上是将特征变换到一个特定的空间中,然后利用数学性质来简化内积运算。

在上面 2 阶多项式 kernel 的基础上,可以进一步配方得到平方形式的 kernel:

$$\Phi(\mathbf{x}) = \left ( \zeta, \sqrt{2\gamma\zeta}x_1, \sqrt{2\gamma\zeta}x_2, \dots, \sqrt{2\gamma\zeta}x_d, \gamma x_1x_1, \gamma x_2x_2, \dots, \gamma x_dx_d \right )$$

于是

$$K(\mathbf{x}^T, \mathbf{x}^{\prime}) = (\zeta + \gamma\mathbf{x}^T\mathbf{x}^{\prime})^2$$

调整参数 $\zeta, \gamma$ 可以改变变换后的几何特性,从而实现不同的内积运算,影响 SVM 的距离计算,
由此来达到不同复杂度的分类边界和 margin。

同样的还可以得出 degree-Q polynomial kernel:

$$K(\mathbf{x}^T, \mathbf{x}^{\prime}) = (\zeta + \gamma\mathbf{x}^T\mathbf{x}^{\prime})^Q$$


\subsection{Gaussian-RBF Kernel}
另一个常用的 kernel 是 Gaussian-RBF kernel:

$$K(\mathbf{x}^T, \mathbf{x}^{\prime}) = \exp \left ( -\gamma \| \mathbf{x} - \mathbf{x}^{\prime} \|^2 \right ), \quad \gamma > 0$$

这个东西是哪两个变换后向量的内积呢?以一维为例:

\begin{equation}
\begin{aligned}
K(x, x^{\prime}) &= \exp \left ( -\| x - x^{\prime} \|^2  \right ) \\
                 &= \exp \left ( -x^2 \right ) \exp \left ( 2xx^{\prime} \right ) \exp \left ( -(x^{\prime})^2 \right ) \\
                 &= \exp \left ( -x^2 \right ) \left ( \sum_{k=0}^{\infty}\frac{2^k(x)^k(x^{\prime})^k}{k!} \right ) \exp \left ( -(x^{\prime})^2 \right )
\end{aligned}
\end{equation}

最后一步用了原点的 Taylor 展开。把这个拆开就得到特征变换:

$$ \Phi(x) = \exp(-x^2) \left ( 1, \sqrt{\frac{2^1}{1!}}x, \sqrt{\frac{2^2}{2!}}x, \dots \right )$$

这玩意儿直接变到了无限维,增加了模型的表达能力,但仍然能算出变换后的内积。

Gaussian-RBF kernel 用的是 Gaussian 函数,这个玩意儿实际上就是正态分布的那种钟形曲线。
最后 SVM 得到的 $g$ 里面,用这个换掉内积,得到的实际上是一堆以支持向量为中心的钟形曲线的线性组合。
参数 $\gamma$ 控制钟的宽度,$\gamma$ 越大越窄(变为一个个尖峰)。

\subsection{Kernel 的选择}
先尝试 linear kernel,即多项式 kernel 中 $\gamma = 1, \zeta = 0, Q = 1$,
也就是恒等变换,直接在原空间中算内积。Linear kernel 的好处是因为没有复杂的变换,
所以可以在得到 $\mathbf{w}$ 之后直接看出不同特征的权重分配结果,有助于数据分析。
但坏处显然是模型表达能力不够,无法做出复杂的分类边界。

多项式 kernel 表达能力强了一些,但太高次多项式 kernel 计算是会有数值计算的误差问题:
要么数值太大表示不了/存不下,要么数值太小几乎就是 0,所以多项式 kernel 一般用
$Q \le 10$ 的次数,同时要慎重选择才 $\zeta, \gamma$。

Gaussian-RBF kernel 表达能力强,但无法像 linear kernel 那样提供对分类结果的解释。
解释说变换到无限维再算内积有点邪乎。

除了这些 kernel 也有别的选择,甚至可以自己构造新 kernel。但必须满足

$$K = \begin{bmatrix}
K(\mathbf{x}_1, \mathbf{x}_1) & K(\mathbf{x}_1, \mathbf{x}_2) & \cdots & K(\mathbf{x}_1, \mathbf{x}_N) \\
K(\mathbf{x}_2, \mathbf{x}_1) & K(\mathbf{x}_2, \mathbf{x}_2) & \cdots & K(\mathbf{x}_2, \mathbf{x}_N) \\
\vdots & \vdots & \vdots & \vdots \\
K(\mathbf{x}_N, \mathbf{x}_1) & K(\mathbf{x}_N, \mathbf{x}_2) & \cdots & K(\mathbf{x}_N, \mathbf{x}_N) \\
\end{bmatrix}$$

这个矩阵必须是半正定的。这个条件是 kernel 合法性的充要条件,称为 Mercer's condition。

\subsection{Soft-margin SVM}
特征变换 + kernel 仍然有过拟合的可能。有时候数据中存在一些 outliers,
直接上复杂的 kernel 很容易过拟合。所以 soft-margin SVM 的目的是要让 SVM
对 outlier 的容忍度,允许一些数据进入 margin 甚至允许一些分类错误,
从而提高对过拟合的抵抗力。

对每个点 $(\mathbf{x}_n, y_n)$,定义 margin violation $\xi_n \ge 0$,
要求

$$y_n (\mathbf{w}^T \mathbf{x}_n + b) \ge 1 - \xi_n$$

也就是说 $\xi_n$ 描述了一个点突破 margin 的程度。为了衡量整个 SVM 对
突破 margin 的容忍性,将所有 $\xi_n$ 求和作为惩罚项加入目标函数。于是
SVM 的问题变为

\begin{equation}
\begin{aligned}
\max_{b, \mathbf{w}, \mathbf{\xi}} &\quad \frac{1}{2} \mathbf{w}^T\mathbf{w} + C \sum_{n=1}^{N}\xi_n \\
\text{subject to} &\quad y_n(\mathbf{w}^{T}\mathbf{x}_n + b) \ge 1 - \xi_n \quad \text{for } n = 1, 2, \dots, N \\
                  &\quad \xi_n \ge 0 \quad \text{for } n = 1, 2, \dots, N
\end{aligned}
\end{equation}

惩罚项的权重 $C$ 表示了我们在允许突破 margin 和要求 margin 最大之间的权衡。
$C$ 越大越不允许越界,也就越接近 hard-margin SVM。$C$ 越小越能容忍越界,
但关心 margin 大。

此时的 Lagrange 函数为

$$\mathcal{L}(b, \mathbf{w}, \bm{\xi}, \bm{\alpha}, \bm{\beta}) = \frac{1}{2}\mathbf{w}^T\mathbf{w} + C \sum_{n=1}^{N}\xi_n + \sum_{n=1}^{N}\alpha_n(1 - \xi_n - y_n(\mathbf{w}^T\mathbf{x}_n + b)) - \sum_{n=1}^{N}\beta_n\xi_n$$

对 $\xi_n$ 求 $\partial\mathcal{L}/\partial\xi_n = 0$ 即得 $C - \alpha_n - \beta_n = 0$,
所以可以把 $\beta_n = C - \alpha_n$ 代回原式子,消掉 $\beta_n$,即

\begin{equation}
\begin{aligned}
\mathcal{L}(b, \mathbf{w}, \bm{\xi}, \bm{\alpha}, \bm{\beta}) &= \frac{1}{2}\mathbf{w}^T\mathbf{w} + C \sum_{n=1}^{N}\xi_n + \sum_{n=1}^{N}\alpha_n(1 - \xi_n - y_n(\mathbf{w}^T\mathbf{x}_n + b)) - \sum_{n=1}^{N}(C - \alpha_n)\xi_n \\
                                                              &= \frac{1}{2}\mathbf{w}^T\mathbf{w} + \sum_{n=1}^{N}\alpha_n(1 - y_n(\mathbf{w}^T\mathbf{x}_n + b))
\end{aligned}
\end{equation}

现在这个问题与 hard-margin SVM 一样了,只是每个 $\alpha_n$ 多了一个上界 $C$:

$$0 \le \alpha_n \le C$$

所以可以直接解出 $\bm{\alpha}$,然后得到 $\mathbf{w}$。但求 $b$ 有所不同。
根据 KKT 条件中的 complementary slackness,在 soft-margin SVM 中要满足

\begin{equation}
\begin{aligned}
\alpha_n^{*} \left ( y_n(\mathbf{w}^T\mathbf{x}_n + b)) - 1 + \xi_n^{*} \right ) &= 0 \\
\beta_n^{*} \xi_n^{*} = (C - \alpha_n^{*}) \xi_n^{*} &= 0
\end{aligned}
\end{equation}

分三种情况讨论下
\begin{enumerate}
  \item $\alpha_n^{*} = 0$,对应的点不是支持向量 \\
  \item $0 < \alpha_n^{*} < C$,根据上面第二个式子得到 $\xi_n^{*} = 0$,代入第一个得到 $y_n(\mathbf{w}^T\mathbf{x}_n + b)) = 1$,
        对应的点是支持向量,称为 free support vector,按 hard-margin SVM 同样的方法解出 $b$ \\
  \item $\alpha_n^{*} = C$,称为 bound support vector,此时 $\xi_n^{*} \ge 0$,对应的点有可能违反 margin 边界。
        此时 $y_n(\mathbf{w}^T\mathbf{x}_n + b)) = 1 - \xi_n^{*} \le 1$,得到的是 $b$ 的取值范围,随便选一个 \\
\end{enumerate}

\subsection{模型选择}
引入 kernel 之后涉及到调参的问题,那么如何选择模型?Validation!

例如对 soft-margin SVM,与参数相关的 $E_{CV}(C, \gamma)$ 很难直接优化,
所以可以做一系列的 $C, \gamma$ 参数组合,然后根据交叉验证的结果来选 $E_{CV}$ 最小的参数组合。

另外,因为 SVM 的 leave-one-out 交叉验证 error 上界受支持向量数的限制,
所以可以通过取不同的参数组合,根据支持向量数来排除一些复杂的模型。
但由于只是上界,所以一般只是作为 $E_{CV}$ 计算代价太高时的 safety check 的方法。


\section{Week 3}
这一周的主要内容就是建立 SVM 和其他线性模型的联系,用 SVM 来做其他事情。
SVM 是 QP,可以转为 dual 再解,也可以用 kernel,所以我们希望把这些特性
也扩展到其他线性模型中。

\subsection{Soft-margin SVM 与 regularization 的联系}
把 $\xi_n$ 看做 $\xi_n = max( 1 - y_n(\mathbf{w}^T\mathbf{z}_n + b), 0 )$,
即点 $\mathbf{z}_n$ 超越分类边界的程度。如果没有超越,根据 SVM 的条件有 
$y_n(\mathbf{w}^T\mathbf{z}_n + b) \ge 1 $,此时 $\xi_n = 0$。这样改写之后,
soft-margin SVM 问题变为

$$\min_{b, \mathbf{w}} \frac{1}{2} \mathbf{w}^T\mathbf{w} + C \sum_{n=1}^{N}max\{1 - y_n(\mathbf{w}^T\mathbf{z}_n + b, 0\}$$

后面那一坨相当于一个 error function,而前一项则可看做 L2 regularization(weight decay)。

为啥不直接解这个 augmented error 的最小化问题?因为 max 形式不好,有不可微点,
而且此问题非 QP,无法转为 dual,无法使用 kernel。

注意这里的常数 $C$ 越大,则 error 占的比例越大,所以是越倾向于减少错误而放宽惩罚项,
对应于普通 L2 regularization 问题 $\displaystyle \frac{\lambda}{N}\mathbf{w}^T\mathbf{w} + E_{in}$
中比较小的 $\lambda$。

\subsection{SVM 的 error function}
令 $s = \mathbf{w}^T\mathbf{z}_n + b$,从前面的式子可以得出 soft-margin SVM 使用的 error function 为

$$\widehat{err_{SVM}}(s, y) = max(1 - ys, 0)$$

也叫 hinge error measure,它是 0-1 error $err_{0/1} = \mathbf{1}\{ys \le 0\}$ 的上界,
而且当 $ys \ge 1$ 时,和 logistic 回归用的 scaled cross entropy $err_{SCE} = \log_2(1 + e^{-ys})$
比较接近。但在 $ys < 0$ 时,hinge error 与 SCE 一样相对于 0/1 error 都是比较松的上界。
基本上可以把 soft-margin SVM 看做带 L2 regularization 的 logistic 回归。


\subsection{用 SVM 做 soft binary classification}
采用一种两阶段学习的方法,即,先通过 SVM 训练出 $\mathbf{w}_{SVM}$ 和 $b_{SVM}$,
然后对这个分类平面求出的分数值进行 scaling 和 shifting,再传入 sigmoid 函数得到概率,即

$$g(\mathbf{x}) = \theta \left ( A(\mathbf{w}_{SVM}^T\Phi(\mathbf{x}) + b_{SVM}) + B\right )$$

问题转化为

$$\operatorname*{min}_{A, B} \frac{1}{N}\sum_{n=1}^{N} log_2 \left ( 1 + \exp \left ( -y_n \left ( A(\mathbf{w}_{SVM}^T\Phi(\mathbf{x}) + b_{SVM}) + B \right ) \right ) \right )$$

解的步骤为
\begin{enumerate}
  \item 执行 SVM 得到$\mathbf{w}_{SVM}$ 和 $b_{SVM}$
  \item 将数据变换到 $\mathbf{z}_n^{\prime} = \mathbf{w}_{SVM}^T\Phi(\mathbf{x_n}) + b_{SVM}$
  \item 执行 logistic regression 得到 $A, B$
  \item 返回 $g$
\end{enumerate}

第三步直接梯度下降或随机梯度下降即可。


\subsection{直接在 $\mathcal{Z}$ 空间解 logistic 回归}
既然 kernel SVM 可以直接在变换后的空间中解问题,而前面说到
SVM 和 logistic 回归又有联系,那么能否用 kernel 直接在变换后的空间中
解 logistic 回归呢?

Representer Theorem:任意 L2-regularized linear model

$$\operatorname*{min}_{\mathbf{w}} \frac{\lambda}{N}\mathbf{w}^T\mathbf{w} + \frac{1}{N}\sum_{n=1}^{N}err(y_n, \mathbf{w}^T\mathbf{z}_n)$$

的最优解 $\mathbf{w}_{*}$ 都可以被表示为数据的线性组合,即 $\displaystyle \mathbf{w}_{*} = \sum_{n=1}^{N}\beta_n\mathbf{z}_n$

证明:$\mathbf{w}_{*}$ 能被表示为数据的线性组合即相当于它在 $span\{\mathbf{z}_n\}$。
假设它不能被表示为线性组合,则可以将其分别正交投影到 $span\{\mathbf{z}_n\}$ 和与它正交的空间,即

$$\mathbf{w}_{*} = \mathbf{w}_{\parallel} + \mathbf{w}_{\perp}$$

所以,error 项

$$err(y_n, \mathbf{w}_{*}^T\mathbf{z}_n) = err(y_n, (\mathbf{w}_{\parallel} + \mathbf{w}_{\perp})^T\mathbf{z}_n)$$

惩罚项

$$\mathbf{w}_{*}^T\mathbf{w}_{*} = \mathbf{w}_{\parallel}^T\mathbf{w}_{\parallel} + 2\mathbf{w}_{\parallel}^T\mathbf{w}_{\perp} + \mathbf{w}_{\perp}^T\mathbf{w}_{\perp} > \mathbf{w}_{\parallel}^T\mathbf{w}_{\parallel}$$

表明 $\mathbf{w}_{\parallel}^T\mathbf{w}_{\parallel}$ 更小,与 $\mathbf{w}_{*}$ 是最优解矛盾。

所以,根据 Representer 定理,可以直接将 $\displaystyle \mathbf{w}_{*} = \sum_{n=1}^{N}\beta_n\mathbf{z}_n$ 代入 L2-regularized logistic 回归问题

$$\operatorname*{min}_{\mathbf{w}} \frac{\lambda}{N}\mathbf{w}^T\mathbf{w} + \frac{1}{N}\sum_{n=1}^{N}log_2(1 + \exp(-y_n\mathbf{w}^T\mathbf{z}_n))$$

得到 kernel logistic regression(KLR):

$$\operatorname*{min}_{\bm{\beta}} \frac{\lambda}{N}\sum_{n=1}^{N} \sum_{n=1}^{M} \beta_n\beta_mK(\mathbf{x}_n\mathbf{x}_m) + \frac{1}{N}\sum_{n=1}^{N}log_2 \left ( 1 + \exp \left ( -y_n \sum_{m=1}^{N}\beta_mK(\mathbf{x}_m, \mathbf{x}_n) \right )  \right )$$

此时只有一个变量,直接 GD、SGD 求解即可。

上面这个式子也可以把前一项看做惩罚项,后一项中的 kernel 则相当于把数据 $\mathbf{x}_n$ 变换到了 $(K(\mathbf{x}_1, \mathbf{x}_n), \dots, K(\mathbf{x}_N, \mathbf{x}_n))$,
并且用 $\bm{\beta}$ 而非 $\mathbf{w}$ 来做权重。

与 SVM 不同的是,直接解这个东西,解出来 $\beta$ 很多都不是 0,所以计算和存储的消耗还是要考虑进去的。


\subsection{Kernel Ridge Regression}
既然 logistic 回归可以用 kernel,那么一般线性回归呢?

采用跟上面类似的方法,根据 Representer 定理,将数据点的线性组合和平方误差函数
代入

$$\operatorname*{min}_{\mathbf{w}} \frac{\lambda}{N}\mathbf{w}^T\mathbf{w} + \frac{1}{N}\sum_{n=1}^{N}(y_n - \mathbf{w}^T\mathbf{z}_n)^2$$

得到

$$\operatorname*{min}_{\bm{\beta}} \frac{\lambda}{N}\sum_{n=1}^{N} \sum_{n=1}^{M} \beta_n\beta_mK(\mathbf{x}_n\mathbf{x}_m) + \frac{1}{N}\sum_{n=1}^{N} \left ( y_n - \sum_{m=1}^{N}\beta_mK(\mathbf{x}_m, \mathbf{x}_n) \right )$$

写成矩阵形式即

$$E_{aug}(\bm{\beta}) = \frac{\lambda}{N}\bm{\beta}^T\mathbf{K}\bm{\beta} + \frac{1}{N}(\bm{\beta}^T\mathbf{K}^T\mathbf{K}\bm{\beta} - 2\bm{\beta}^T\mathbf{K}^T\mathbf{y} + \mathbf{y}^T\mathbf{y})$$

求梯度

$$\nabla E_{aug}(\bm{\beta}) = \frac{2}{N} \mathbf{K}^T ((\lambda\mathbf{I} + \mathbf{K})\bm{\beta} - \mathbf{y})$$

直接解得

$$\bm{\beta} = (\lambda\mathbf{I} + \mathbf{K})^{-1}\mathbf{y}$$

这个计算量与数据量相关,而普通的回归问题与数据维度相关。同时,
解出来 $\beta$ 也有很多不为 0。所以这里需要权衡,普通回归 $N \gg d$ 时比较好用,
$N$ 一大,kernel 回归效率就低了。但同时普通回归的表达能力不如 kernel 回归。

注意到,这里实际上是包含了转换前数据中的常数项 bias。或者可以从普通的 ridge regression 推导出 kernel ridge regression。
普通回归即

$$ \hat{\mathbf{y}} = \mathbf{X}\bm{\beta}, \quad \bm{\beta} = (\mathbf{X}^T\mathbf{X} + \lambda\mathbf{I})^{-1}\mathbf{X}^T\mathbf{y}$$

根据 

$$(\mathbf{X}^T\mathbf{X} + \lambda\mathbf{I})\mathbf{X}^T = \mathbf{X}^T\mathbf{X}\mathbf{X}^T + \lambda\mathbf{X}^T = \mathbf{X}^T(\mathbf{X}\mathbf{X}^T + \lambda\mathbf{I})$$

左边乘 $(\mathbf{X}^T\mathbf{X} + \lambda\mathbf{I})^{-1}$ ,右边乘 $(\mathbf{XX^T + \lambda\mathbf{I}})^{-1}\mathbf{y}$,
得到

$$\mathbf{X}^T(\mathbf{XX}^T + \lambda\mathbf{I})^{-1}\mathbf{y} = (\mathbf{X}^T\mathbf{X} + \lambda\mathbf{I})^{-1}\mathbf{X}^T\mathbf{y} = \bm{\beta}$$

令 $\bm{\alpha} = (\bm{XX}^T + \lambda\mathbf{I})^{-1}\mathbf{y}$,则

$$\bm{\beta} = \mathbf{X}^T\bm{\alpha}$$

对于一个要预测的数据点 $\mathbf{x}$,有

$$\hat{y} = \bm{\beta}^T\mathbf{x} = \mathbf{y}^T(\mathbf{XX}^T + \lambda\mathbf{I})^{-1}\mathbf{Xx}$$

注意到,这里的 $\mathbf{XX}^T$ 即训练数据中所有的点之间的内积,而 $\mathbf{Xx}$ 则是每个训练数据与待预测数据的内积。
定义前者为矩阵 $\mathbf{K}$,后者为向量 $\mathbf{k}$,则上式可改写为

$$\hat{y} = \bm{\beta}^T\mathbf{x} = \mathbf{y}^T(\mathbf{K} + \lambda\mathbf{I})^{-1}\mathbf{k}$$

这里都是基于 ridge regression 推出来的,所以都是默认将训练数据增加一个分量 1,对应到权重的常数项 bias。
如果做了特征变换,则相当于保持补充的分量 1 不变,其他维度输入给转换函数 $\Phi$,然后这里的内积就可以换成
kernel。


\subsection{用 kernel ridge regression 做分类}
此方法即 least-square SVM(LSSVM),和普通 SVM 得到的分类边界差不多,但支持向量更多,
得到的非零结果也更多。

\subsection{Support Vector Regression}
我觉得这个方法才是真正结合了 SVM 思想的回归。
思路时在回归出来的函数上下加上 margin,如果数据点在 margin 内则认为没有回归误差,
否则计算误差。与 soft-margin SVM 一样,相当于提高了模型的容忍度。

这里用到的 error 函数为

$$err(y, s) = max(0, |s-y| - \epsilon)$$

其中 $\epsilon$ 为 margin 宽度,如果模型计算出的分值 $s$ 与正确值 $y$ 的差距小于 margin 宽度
则 error 为 0,否则取这个差值作为 error。

注意这个 tube error 实际上和平方误差在 $|s-y|$ 较小时是很接近的,而越往两边走,
平方误差增长越快。所以 tube error 更少受到 outliers 的影响,因为平方误差在这种情况下
通常值比较大,模型会更多地去做修正,从而容易对 outlier 产生过拟合。

仿照 soft-margin SVM,可以构造出 tube regression 的问题:

$$\operatorname*{min}_{b, \mathbf{w}}\frac{1}{2}\mathbf{w}^T\mathbf{w} + C\sum_{n=1}^{N} \operatorname*{max} (0, |\mathbf{w}^T\mathbf{z}_n + b - y_n| - \epsilon)$$

然后替换掉上面那一坨 max,把这个东西转换为 QP,史称 support vector regression(SVR)。

\begin{equation}
\begin{aligned}
\min_{b, \mathbf{w}, \bm{\xi}} \quad & \frac{1}{2}\mathbf{w}^{T}\mathbf{w} + C \sum_{n=1}^{N}\xi_n \\
\text{subject to} \quad & |\mathbf{w}^{T}\mathbf{z}_n + b - y_n| \ge \epsilon + \xi_n \\
& \xi_n \ge 0 \\
& n = 1, 2, \dots, N \\
\end{aligned}
\end{equation}

现在的问题是如何确定 $\xi_n$?$\xi_n$ 实际上是在衡量预测分数与实际值之差是否有超过 margin,或者超过了多少。
所以上面的约束条件规定的是每个点 $\mathbf{z}_n$ 在 $\xi_n$ 的容忍度之下要满足的范围。
根据绝对值里面那一坨的符号,可以把容忍度 $\xi_n$ 分为在回归线上面和下面的容忍度:$\xi_n^{\vee}, \xi_n^{\wedge}$,
所以可以将上面的问题改写为:

\begin{equation}
\begin{aligned}
\min_{b, \mathbf{w}, \bm{\xi}^{\vee}, \bm{\xi}^{\wedge}} \quad & \frac{1}{2}\mathbf{w}^{T}\mathbf{w} + C \sum_{n=1}^{N}(\xi_n^{\vee} + \xi_n^{\wedge}) \\
\text{subject to} \quad & -\epsilon - \xi_n^{\vee} \le y_n - \mathbf{w}^{T}\mathbf{z}_n - b \le \epsilon + \xi_n^{\wedge} \\
& \xi_n^{\vee} \ge 0, \xi_n^{\wedge} \ge 0 \\
& n = 1, 2, \dots, N \\
\end{aligned}
\end{equation}

此问题总共有 $\tilde{d} + 1 + 2N$ 个变量,$2N + 2N$ 个约束。

将此问题转化为 dual,分别设置 $\xi_n^{\vee}$ 和 $\xi_n^{\wedge}$ 的 lagrange multiplier 为
$\alpha_n^{\vee}$ 和 $\alpha_n^{\wedge}$,得到 dual

\begin{equation}
\begin{aligned}
\min_{\bm{\xi}^{\vee}, \bm{\xi}^{\wedge}} \quad & \frac{1}{2}\sum_{n=1}^{N}\sum_{m=1}^{N}(\alpha_n^{\wedge} - \alpha_n^{\vee})(\alpha_m^{\wedge} - \alpha_m^{\vee})K(\mathbf{x}_n, \mathbf{x}_m) \\
& + \sum_{n=1}^{N}((\epsilon - y_n)\alpha_n^{\wedge} + (\epsilon + y_n)\alpha_n^{\vee}) \\
\text{subject to} \quad & \sum_{n=1}^{N}(\alpha_n^{\wedge} - \alpha_n^{\vee}) = 0 \\
& 0 \le \alpha_n^{\wedge} \le C, 0 \le \alpha_n^{\vee} \le C \\
& n = 1, 2, \dots, N \\
\end{aligned}
\end{equation}

SVR 的解是否稀疏呢?
根据前面可以推出

$$\mathbf{w} = \sum_{n=1}^{N}(\alpha_n^{\wedge} - \alpha_n^{\vee})\mathbf{z}_n$$

根据 KKT 之 complementary slackness

\begin{equation}
\begin{aligned}
\alpha_n^{\wedge}(\epsilon + \xi_n^{\wedge} - y_n + \mathbf{w}^T\mathbf{z}_n + b) &= 0 \\
\alpha_n^{\vee}(\epsilon + \xi_n^{\vee} + y_n - \mathbf{w}^T\mathbf{z}_n - b) &= 0 \\
\end{aligned}
\end{equation}

当 $|\mathbf{w}^T\mathbf{z}_n + b - y_n| < \epsilon$ 时,误差可以被忽略,所以 $\xi_n^{\wedge} = \xi_n^{\vee} = 0$,
而且 $(\epsilon + \xi_n^{\wedge} - y_n + \mathbf{w}^T\mathbf{z}_n + b) \ne 0$,
$(\epsilon + \xi_n^{\vee} + y_n - \mathbf{w}^T\mathbf{z}_n - b) \ne 0$,
所以 $\alpha_n^{\wedge} = \alpha_n^{\vee} = 0$,对应的 $\mathbf{z}_n$ 系数为 0,非支持向量,不贡献给 $\mathbf{w}$。

所以 SVR 的解是稀疏的。


\subsection{模型比较}

\begin{tabular}{p{3cm}|p{3cm}|p{3cm}|p{3cm}}
\hline
算法 & 问题形式 & 解法 & Error Measure \\ 
\hline
\hline
\hline
PLA/Pocket           & min     & iteration            & $err_{0/1}$ \\
\hline
SVR                  & min+reg & QP on primal         & $err_{TUBE}$ \\
\hline
Soft-margin SVM      & min+reg. & QP on primal        & $\widehat{err_{SVM}}$ \\
\hline
Ridge Regression     & min+reg. & analytical solution & $err_{SQR}$ \\
\hline
Logistic Regression  & min+reg. & GD/SGD              & $err_{SCE}$ \\
\hline
SVM                  & min      & QP on dual          & \\
\hline
Dual SVR             & min      & QP on dual          & \\
\hline
probabilistic SVM    & min      & SVM then logistic   & \\
\hline
Kernel Ridge Reg.    & min+reg. & kernel + analytical & \\
\hline
Kernel Log. Reg.     & min+reg. & kernel + GD/SGD     & \\
\hline
\end{tabular}


\section{Week 4}
\subsection{Aggregation}
Aggregation 的思路是,手里有一堆(通常比较弱的)hypothesis $g_1, g_2, \dots, g_T$ 时,通过投票的方法得到一个综合所有 $g$ 的更强的 $G$。

常用的投票方法
\begin{enumerate}
  \item 根据 validation error 选择最优 $g$:$\displaystyle G(\mathbf{x}) = g_{k}(\mathbf{x}), k = \operatorname*{argmin}_{t \in {1, 2, \dots, T}}E_{val}(g_t^{-})$ \\
  \item 均匀混合:$G(\mathbf{x}) = sign(\sum_{t=1}^{T}g_t(\mathbf{x}))$ \\
  \item 非均匀混合:$G(\mathbf{x}) = sign(\sum_{t=1}^{T}\alpha_tg_t(\mathbf{x})), \alpha_t \ge 0$。
        这个方法包含前两种,第一种相当于取 $\alpha_t = \mathbf{1}\{E_{val}(g_t^{-}) \text{最小}\}$,
        第二种相当于取 $\alpha_t = 1$ \\
  \item 按条件混合:$G(\mathbf{x}) = sign(\sum_{t=1}^{T}q_t(\mathbf{x})g_t(\mathbf{x})), q_t(\mathbf{x}) \ge 0$。
        显然取 $q_t(\mathbf{x}) = \alpha_t$ 就可以包含第三种了
\end{enumerate}

两个例子:通过 decision stump 获得比较复杂的分类边界(类似于特征变换);通过 PLA 得到的(随机的)分类线得到一个平均的分类线(类似于 SVM 的最大 margin regularization)。


\subsection{Uniform Blending}
如前所述,每人一票:

$$G(\mathbf{x}) = sign(\sum_{t=1}^{T}g_t(\mathbf{x}))$$

多类问题中,投票改成哪个类别被投得多就选哪个:

$$G(\mathbf{x}) = \operatorname*{argmax}_{1 \le k \le K}\sum_{t=1}^{T}\mathbf{1}\{g_t(\mathbf{x}) = k\}$$

回归问题中,改为计算每个 $g$ 输出值的平均:

$$G(\mathbf{x}) = \frac{1}{T}\sum_{t=1}^{T}g_t(\mathbf{x})$$

根据回归问题可以推一个上界。假设上面取平均的操作为 $avg$,目标函数为 $f$,则对于某个 $\mathbf{x}_n$(略去不写了),有

\begin{equation}
\begin{aligned}
avg((g_t - f^2)) &= avg(g_t^2 - 2g_tf + f^2) \\
                 &= avg(g_t^2) - 2Gf + f^2 \\
                 &= avg(g_t^2) - G^2 + (G - f)^2 \\
                 &= avg(g_t^2) - 2G^2 + G^2 + (G - f)^2 \\
                 &= avg(g_t^2 - 2g_tG + G^2) + (G - f)^2 \\
                 &= avg((g_t - G)^2) + (G - f)^2 \\
\end{aligned}
\end{equation}

将上式两边对所有 $\mathbf{x}_n$ 求和/积分,就得到上面左边即随便选一个 $g_t$ 与 $f$ 的误差 $E_{out}(g_t)$,右边第二项即 $G$ 的误差 $E_{out}(G)$,
右边第一项即 $avg(\mathbb{E}((g_t - G)^2)$,所以

$$avg(E_{out}(g_t)) \ge E_{out}(G)$$

假设我们通过这样的方式来学习到最终的 hypothesis:
\begin{enumerate}
  \item 从某分布 i.i.d 得到一组大小为 $N$ 的数据 \\
  \item 通过算法 $\mathcal{A}$ 得到 $g_t$ \\
  \item 一共进行 $T$ 轮这样的步骤,最后求所有 $g_t$ 的平均在 $T \rightarrow \infty$ 时的极限 $\bar{g}$ \\
\end{enumerate}

最后按上面的式子可得

$$E_{out}(g_t) = avg(\mathbb{E}((g_t - \bar{g})^2)) + E_{out}(\bar{g})$$

上式说明:算法 $\mathcal{A}$ 的表现等于右边第一项:variance,加上右边第二项:bias。
其中 bias 是 consensus $\bar{g}$ 的表现。

Uniform blending 实际上就是在减小 variance 来获得更稳定的表现。


\subsection{Linear Blending}
按权重投票:

$$G(\mathbf{x}) = sign(\sum_{t=1}^{T}\alpha_tg_t(\mathbf{x})), \alpha_t \ge 0$$

问题变为了找出使 $E_in$(一般当然要 validation 了,用 $E_{val}$) 最小的 $\bm{\alpha}$。

对于线性回归,即

$$\operatorname*{min}_{\alpha_t \ge 0}\frac{1}{N}\sum_{n=1}^{N} \left ( y_n - \sum_{t=1}^{T}\alpha_tg_t(\mathbf{x}) \right )^2$$

这东西就是一个线性模型,相当于是用每个 $g_t$ 给每个 $\mathbf{x}_n$ 做特征转换为 ${g_1(\mathbf{x}), g_2(\mathbf{x}), \dots, g_T(\mathbf{x})}$,再用 $\alpha_t$ 来加权,
唯一的区别是这里权值 $\alpha_t$ 都有约束。但这里的约束通常可以忽略掉,因为对于分类来说,如果 $\alpha_t < 0$,你给它
加个绝对值,把符号搞到后面,就变成 $-g_t(\mathbf{x})$,相当于预测结果取了个反,得到的分类器还是一样的。(但是对于其他问题呢?)

\subsection{Any Blending}
Any Blending 即把 linear blending 中的线性模型换成任意模型。注意如果做了交叉验证,
训练时用的相当于 ${g_1^{-}(\mathbf{x}), g_2^{-}(\mathbf{x}), \dots, g_T^{-}(\mathbf{x})}$,
最后返回最终结果时,要将数据变为 ${g_1(\mathbf{x}), g_2(\mathbf{x}), \dots, g_T(\mathbf{x})}$。

Any blending 可以实现 conditional blending,即先判断 $\mathbf{x}$ 是否符合某种条件,然后再用 $g$ 预测。


\subsection{Bagging: Bootstrap Aggregation}
之前的各种 blending 手段前提都是手上已经有了一堆的 $g$,
那么 能不能一边得到参与投票的新  $g$,一边做 aggregate?

要得到新的 $g$,就要通过各种方式在现有 $g$ 的基础上产生差异,如
\begin{enumerate}
  \item 来自不同模型的 $g$ \\
  \item 同一模型的不同参数,如 GD \\
  \item 算法本身的随机性,如 PLA \\
  \item 数据的随机性,如交叉验证 \\
\end{enumerate}

或者,根据前面推导的那个上界,在同一个算法的基础上通过不同的数据来得到新的 $g$,
但又必须保证和原训练数据来自同一分布。于是可以用 bootstrap 的方法,从原训练数据中
采样获得新的数据——有放回地从 $\mathcal{D}$ 抽取 $N^{\prime}$ 个数据。

Bagging 特别适合于对数据随机性敏感的 $g$。


\subsection{Adaptive Boosting (AdaBoost)}
AdaBoost 的目的是让若干较弱的分类器 $g$ 组合成一个更强的分类器 $G$。

从 bootstrap 出发,每一轮做过抽样之后,每组数据 $\mathbf{x}$ 相当于有了一个权值,
相当于 $\mathbf{x}$ 在抽样出来的数据集中占有多大比例。而我们要最小化的误差函数
则相应地算入了 $\mathbf{x}$ 的权值。

回忆教小孩认苹果的例子,在 bagging 过程中,通过人为地设置权重,可以起到类似于老师那种
强调犯错的例子、弱化正确例子的作用,这样做的理由是增加 $g$ 的差异性,让每个 $g$ 专注于
某个方面的“专长”。

具体到算法中,如果第 $t$ 轮的 $g_t$ 在第 $t+1$ 轮的权重分配 $u_n^{(t+1)}$ 下表现很糟糕,
那么显然第 $t+1$ 轮基本不会得到 $g_t$ 或者和它类似的 hypothesis,从而让 $g_{t+1}$ 产生了和 $g_t$
的差异性。

要让 $g_t$ 表现糟糕,相当于让它的表现与随机乱分差不多,即在第 $t+1$ 轮有

$$\frac{\sum_{n=1}^{N}u_n^{(t+1)}\mathbf{1}\{y_n \ne g_t(\mathbf{x}_n)\}}{\sum_{n=1}^{N}u_n^{(t+1)}} = \frac{1}{2}$$

那么如何调整呢?简单推算即可得到,从第 $t$ 轮到 $t+1$ 轮,只需要
\begin{itemize}
  \item 分错的数据权值调整为正比于 $1 - \epsilon_t$
  \item 分对的数据权值调整为正比于 $\epsilon_t$
\end{itemize}

其中 $\epsilon_t$ 为 $g_t$ 在第 $t$ 轮中的错误率。

这两个 scaling 操作可以统一为一个因子 $\displaystyle e_t = \sqrt{\frac{1 - \epsilon_t}{\epsilon_t}}$,
每轮更新样本权重时,分错的样本乘以 $e_t$ ,分对的除以 $e_t$。这个因子的物理意义是,如果 $\displaystyle \epsilon_t \le \frac{1}{2}$,
则 $e_t \ge 1$。也就是说,如果 $g_t$ 的错误率低于随机乱分,即表现优于随机乱分,则将犯错的样本权重增大,分对的样本权重减小,
从而使后面的 $g$ 更多地聚焦在分错的数据上。

最后 aggregation 的过程可以像 bagging 一样用各种线性非线性来搞,也可以用 AdaBoost 的方法来搞,即根据各个弱分类器的表现
来赋予它们权重,取 $\displaystyle \alpha_t = \ln e_t = \ln\sqrt{\frac{1 - \epsilon_t}{\epsilon_t}}$,在每轮迭代中直接计算出来,
最后返回 $G(\mathbf{x}) = sign(\sum_{t=1}^{T}\alpha_t g_t(\mathbf{x}))$

初始时,设置所有 $\displaystyle u^{(1)} = \frac{1}{N}$。

AdaBoost 能很快地将 $E_{in}$ 搞到很小。



\section{Week 5}
\subsection{Decision Tree \& CART}
Decision tree、AdaBoost、bagging 都是在一开始没有各种 $g$ 的时候进行学习的方法。
Decision tree 对应于 blending 中的 conditional aggregation,即按条件地做预测。

决策树中每个叶节点相当于一个 $g$,从根节点到叶节点的每条路径都对应这个 $g$ 要符合的条件,
即,从根是否存在一条到这个叶节点的路径

$$G(\mathbf{x}) = \sum_{c=1}^{C}\mathbf{1}\{\mathbf{x}\text{ is on path }t\} \cdot g_t(\mathbf{x})$$

或者可以用递归的角度来看,在每个节点先通过一个分支函数 $b(\mathbf{x})$ 来决定应该
走哪个分支,然后再执行对应分支下的子树的决策函数,即

$$G(\mathbf{x}) = \sum_{c=1}^{C}\mathbf{1}\{b(\mathbf{x}) = c\} \cdot G_c(\mathbf{x})$$

递归的思路比较适合程序实现。

这里主要介绍了 CART,一种比较由代表性的决策树算法:每个 $g$ 都返回常数,构建二叉树。

\begin{enumerate}
  \item 如果不用分支了,返回使 $E_{in}$ 最小的常数 \\
  \item 基于 decision stump 学习节点分支条件 $\displaystyle b(\mathbf{x}) = \operatorname*{argmin}_{h}\sum_{c=1}^{2}|\mathcal{D}_c| \times impurity(\mathcal{D}_c)$,
  其中 $\mathcal{D}_c$ 表示用 decision stump $h$ 分割出来的一部分数据 \\
  \item 将数据分为两部分 $\mathcal{D}_c = \{(\mathbf{x}_n, y_n): b(\mathbf{x}_n) = c\}$ \\
  \item 递归构建子树 $G_c \leftarrow \text{\texttt{DecisionTree}}(\mathcal{D}_c)$ \\
  \item 递归回来返回 $\displaystyle G(\mathbf{x}) = \sum_{c=1}^{2}\mathbf{1}\{b(\mathbf{x} = c)\}G_c(\mathbf{x})$ \\
\end{enumerate}

多类问题时直接修改一下就好。

上面每一个节点都要最小化 impurity,说白了就是要让每一个 decision stump 切出来的两半边,错分数量最小。

Impurity 的计算,回归常用

$$impurity(\mathcal{D}) = \frac{1}{N}\sum_{n=1}^{N}(y_n - y^{*})^2$$

而分类常用 Gini index

$$1 - \sum_{k=1}^{K}\left ( \frac{\sum_{n=1}^{N}\mathbf{1}\{y_n = k\}}{N} \right )^2$$

也就是计算 1 减去每一类分对的数据比例的平方和。

\subsection{Pruning}
根据叶节点数量 $\Omega(G)$ 调整惩罚力度

$$\operatorname*{argmin}_{G}E_{in}(G) + \lambda\Omega(G)$$

通常训练时先得到一棵不剪枝的树 $G^{(0)}$,然后通过合并叶节点的方式依次得到少 1 片叶子 $G^{(1)}$、2 片叶子 $G^{(2)}$……的树,
再通过上面的公式计算,得到一个能最小化 regularized error 的 $G$。$\lambda$ 可通过交叉验证来选择。


\subsection{Decision Tree 总结}
训练中用到的 decision stump 可以换成 decision subset 来处理非数值类型的特征。

训练节点分支时,缺失的特征可以通过 surrogate branch 来解决,即在训练中找出与某个特征具有相似判断标准的特征,
然后遇到缺失特征可以由备用的这些特征来替代。

决策树模型简单,很符合人类做决策的过程,而且有比较直观的可解释性(对比一下 kerne SVM)。

从可视化的角度来看,2D 决策树做出的分类边界中可能存在只划分部分点的分界线(子树),而 AdaBoost 中的所有分类线
都是横穿整个平面。


\subsection{Random Forest}
随机森林结合 bagging 和决策树,即在 bagging 过程中,随机有放回重新抽样 $N^{\prime}$
组数据用于训练决策树,最后用 uniform blending 组合所有决策树。

注意到 bagging 的每一轮迭代中,重新抽样和决策树的训练都是独立的,所以随机森林
可以方便地将每一棵决策树的训练并行化,而且 bagging 也有利于减少单棵决策树过拟合的风险。

除了 bagging 本身的训练数据随机化,还可以引入特征随机化:先抽出 $N^{\prime}$ 组数据,
然后对其中的每个 $\mathbf{x}$ 做一个特征变换 $\Phi(\mathbf{x}) = \mathbf{Px}$,
变换矩阵 $\mathbf{P}$ 相当于选择了原 $\mathbf{x}$ 中的部分特征进行加权组合,将数据
转换到一个(通常是)低维空间中。这两次随机化(bagging + random-combination)使
每棵树的训练产生了更多了差异性。

\subsection{Out-of-bag Data}
Bagging 抽样没有抽中的数据称为 out-of-bag data,某组数据在 $N$ 次抽样中
都没有被抽到的概率为

$$\lim_{N \to \infty}\left ( 1 - \frac{1}{N} \right )^N = \frac{1}{\left ( 1 + \frac{1}{N-1} \right )^N} = \frac{1}{e}$$

所以平均每棵树的训练中都有约 $\displaystyle \frac{1}{e}\cdot N$ 组数据没有被抽中。
这些未抽中数据可以用来对某个 $G^{-}$ 做 validation。对每一组 $(\mathbf{x}_n, y_n)$,
找出没有用这组数据训练出的所有决策树 $g_t$,得到 $G_n^{-} = average(g_t)$,然后
用 $(\mathbf{x}_n, y_n)$ 来 validate $G_n^{-}$。最后将所有的 error 平均,即

$$E_{OOB}(G) = \frac{1}{N}\sum_{n=1}^{N}error(y_n, G_n^{-}(\mathbf{x}_n))$$

这个特性允许我们不用像交叉验证那样做重复的训练,即可得到最终 $G$ 的评估结果。

\subsection{特征选择}
特征选择可以降低数据维度,简化训练过程,也可以去除一些有噪数据,减小过拟合风险,
并增加模型的可解释性。决策树自带特征选择哦。

两种特征选择的方式:
\paragraph{线性模型}
先用全部特征训练一个线性模型,再根据每个特征的权重 $|w_i|$ 选择权重较大的特征用于训练其他模型

\paragraph{Permutation test}
将训练数据中的某个特征 $x_i$ 随机重排,再用重排后的数据训练出新模型,比较前后
两次训练结果的表现差异。差异越大说明此特征越重要。
在随机森林中,可以通过 out-of-bag data 来避免重复训练。方法是在计算 out-of-bag error $E_{OOB}$ 时,
将 out-of-bag data $\mathbf{x}_n$ 中的特征 $x_i$ 随机换为某个其他 out-of-bag data $\mathbf{x}_m$ 的 $x_i$。

\subsection{Random Forest 总结}
随机森林可以通过多棵决策树做出平滑且 margin 较大的边界。基本上树越多越好。
训练的时候最好检查一下树的数量是不是够多,是否让随机性达到了较为稳定的状态。


\section{Week 6}
\subsection{AdaBoosted Decision Tree}
步骤:
\begin{enumerate}
  \item 第 $t=1,2,\dots,T$ 轮迭代 \\
  \item 用 $\mathbf{u}^{(t)}$ 对训练数据加权 \\
  \item 根据权重训练决策树 $g_t$ 为 \texttt{DecisionTree}($\mathcal{D}, \mathbf{u}^{(t)}$) \\
  \item 计算 $g_t$ 的投票权重 $\alpha_t$
  \item 返回 $\mathbf{\alpha}$ 加权投票的 $G$
\end{enumerate}

由于 AdaBoost 中训练每棵决策树时要最小化 $\displaystyle E_{in}^{\mathbf{u}}(h) = \frac{1}{N}\sum_{n=1}^{N}u_n \cdot error(y_n, h(\mathbf{x}_n))$,
我们希望将决策树当做黑盒,而不要修改其中的 error function。
所以改用调整数据的方式,将数据权重 $\mathbf{u}$ 融合进去。说白了就是根据
$\mathbf{u}$ 对数据抽样,抽出大小为 $N^{\prime}$ 的 $\tilde{\mathcal{D}}_t$。

每棵决策树的投票权重 $\displaystyle \alpha_t = \ln{\sqrt{\frac{1-\epsilon_t}{\epsilon_t}}}$,
这个 weighted error rate $\epsilon_t$ 如果在决策树完全长成的情况下会是 0,
这棵树对应的 $\alpha_t$ 算出来会是 $+\infty$,所以需要剪枝,即用较弱的决策树。

如果用前面说到的限制节点数的方法来剪枝,剪到最极端的情形,只剩 1 个点,
就等于在决策树中只需要学习分支函数,也就是一个 decision stump。
所以在二元分类情况下,AdaBoosted Decision Stump 是 AdaBoosted Decision Tree 的特例。


\subsection{AdaBoost as Functional Gradient Descent}
下面主要推导如何将 AdaBoost 作为 functional gradient descent。

首先将 AdaBoost 中的样本权重更新过程统一为

$$u_n^{(t+1)} = u_n^{(t)}\exp(-y_n \alpha g_t(\mathbf{x}_n))$$

做完 $T$ 轮迭代后

$$u_n^{(T+1)} = u_n^{(1)} \cdot \prod_{t=1}^{T}\exp(-y_n \alpha_t g_t(\mathbf{x}_n)) = \frac{1}{N}\exp\left ( -y_n \sum_{t=1}^{T}\alpha_t g_t(\mathbf{x}_n) \right )$$

而最终返回的 $G$ 为

$$G(\mathbf{x}) = sign\left (\sum_{t=1}^{T}\alpha_t g_t(\mathbf{x})  \right )$$

中间的求和即对 $\mathbf{x}$ 算出的分数。

如果把 $g_t(\mathbf{x}_n)$ 看做特征转换,$\alpha_t$ 做权重,联系到 SVM,
$y_n \cdot score$ 相当于计算 margin,我们肯定想让 margin 尽可能大,那就要让 $\exp(-y_n \cdot score)$ 尽可能小,
也就是让上面迭代完之后的 $u_n^{(T+1)}$ 尽可能小。而事实上 AdaBoost 会逐步减小 $\displaystyle \sum_{n=1}^{N}u_n^{(t)}$。
所以将上面 $u_n^{(T+1)}$ 的计算公式代入,就得到

$$\sum_{n=1}^{N}u_n^{(T+1)} = \frac{1}{N}\sum_{n=1}^{N}\exp\left ( -y_n \sum_{t=1}^{T}\alpha_t g_t(\mathbf{x}_n) \right )$$

令 $\displaystyle s = \sum_{t=1}^{T}\alpha_t g_t(\mathbf{x}_n)$,则上面这个式子可看做一个 error function

$$\widehat{err}_{ADA}(s, y) = \exp(-ys)$$

称为 exponential error measure,这个东西也是 0/1 错误的上界。

那么 AdaBoost 如何通过迭代来逐步最小化这个东西?联系梯度下降的迭代,
AdaBoost 相当于是找到一个 functional 的梯度方向 $h(\mathbf{x}_n)$

\begin{equation}
\begin{aligned}
\widehat{E}_{ADA} &= \frac{1}{N}\sum_{n=1}^{N}\exp\left (-y_n \left (\sum_{k=1}^{t-1}\alpha_k g_k(\mathbf{x}_n) + \eta h(\mathbf{x}_n) \right ) \right ) \\
                  &= \sum_{n=1}^{N} u_n^{(t)}\exp(-y_n\eta h(\mathbf{x}_n)) \\
                  &\approx \sum_{n=1}^{N} u_n^{(t)}(1 - y_n\eta h(\mathbf{x}_n)) \\
                  &= \sum_{n=1}^{N} u_n^{(t)} - \eta \sum_{n=1}^{N} u_n^{(t)}y_n h(\mathbf{x}_n) \\
\end{aligned}
\end{equation}

来最小化 $\displaystyle \sum_{n=1}^{N}u_n^{(t)}(-y_n h(\mathbf{x}_n))$。

然后分情况讨论

\begin{equation}
\begin{aligned}
\sum_{n=1}^{N}u_n^{(t)}(-y_n h(\mathbf{x}_n)) &= \sum_{n=1}^{N}u_n^{(t)}
  \begin{cases}
    -1 & \quad \text{if } y_n =   h(\mathbf{x}_n) \\
    +1 & \quad \text{if } y_n \ne h(\mathbf{x}_n) \\
  \end{cases} \\
                                              &= -\sum_{n=1}^{N}u_n^{(t)} + \sum_{n=1}^{N}u_n^{(t)}
  \begin{cases}
    0 & \quad \text{if } y_n =   h(\mathbf{x}_n) \\
    2 & \quad \text{if } y_n \ne h(\mathbf{x}_n) \\
  \end{cases} \\
                                              &= -\sum_{n=1}^{N}u_n^{(t)} + 2N E_{in}^{\mathbf{u}^{(t)}}(h) \\
\end{aligned}
\end{equation}

最后一个等号是因为那个 0、2 的 case 实际算的就是 weighted error。
显然,能够最小化 weighted error 的就是 AdaBoost 的基础算法。
所以 AdaBoost 就是在用基础算法找到这个最佳的梯度方向 $g_t = h$。

找到 $g_t$ 后,$\widehat{E}_{ADA}$ 中要优化的只有后面一项 $\displaystyle \sum_{n=1}^{N}u_n^{(t)}(-y_n g_t(\mathbf{x}_n))$。
然后对于下降的步长 $\eta$,可以用最激进的方式直接找到下降最多的值,称为 steepest descent。
根据 $\exp(\dots)$ 一项的符号可以分情况讨论:

$$
\begin{cases}
  u_n^{(t)}\exp(-\eta) & \quad \text{if } y_n = g_t(\mathbf{x}_n) \\
  u_n^{(t)}\exp(+\eta) & \quad \text{if } y_n \ne g_t(\mathbf{x}_n) \\
\end{cases}
$$

要求最好的 $\eta$ 直接求导即可,计算得到 $\eta_t = \ln{\sqrt{\frac{1-\epsilon_t}{\epsilon_t}}} = \alpha_t$。
所以 AdaBoost 做的实际上就是一个 functional 的 steepest descent。

\subsection{Gradient Boosting \& Gradient Boosted Regression}
在上面的过程中,把 error function 改为任意函数,即从

$$\operatorname*{min}_{\eta} \operatorname*{min}_{h} \frac{1}{N}\sum_{n=1}^{N} \exp\left (-y_n \left (\sum_{k=1}^{t-1}\alpha_k g_k(\mathbf{x}_n) + \eta h(\mathbf{x}_n) \right ) \right )$$

变为

$$\operatorname*{min}_{\eta} \operatorname*{min}_{h} \frac{1}{N}\sum_{n=1}^{N} error \left ( \sum_{k=1}^{t-1}\alpha_k g_k(\mathbf{x}_n) + \eta h(\mathbf{x}_n), y_n \right )$$

就得到 gradient boosting。

特别地,如果用平方错误来做回归,则上面的最小化相当于

$$\operatorname*{min}_{\eta} \operatorname*{min}_{h} \frac{1}{N}\sum_{n=1}^{N} error \left ( s_n + \eta h(\mathbf{x}_n), y_n \right )$$

在 $s_n$ 处泰勒展开,里面根据 $h$ 最小化的部分即

$$\operatorname*{min}_{h} \frac{1}{N}\sum_{n=1}^{N}error(s_n, y_n) + \frac{1}{N}\sum_{n=1}^{N}\eta h(\mathbf{x}_n)\frac{\partial error(s, y_n)}{\partial s}|_{s=s_n}$$

前面一部分都不影响优化操作,最后求和部分即

$$\sum_{n=1}^{N}h(\mathbf{x}_n) \cdot 2(s_n - y_n)$$

整个最小,显然要对 $h(\mathbf{x}_n)$ 加以限制,否则直接取 $h(\mathbf{x}_n) = -\infty \cdot (s_n - y_n)$ 就行了。
类似于 regularization,引入 $h$ 的约束,上式变为最小化

$$\sum_{n=1}^{N} \left ( 2 \cdot h(\mathbf{x}_n) (s_n - y_n) + (h(\mathbf{x}_n))^2 \right )$$

配方得

$$\sum_{n=1}^{N} \left ( constant + (h(\mathbf{x}_n) - (y_n - s_n))^2 \right )$$

其他省略的部分都是与优化操作无关的常量,而后面需要找的最好的 $h(\mathbf{x}_n)$,
就相当于对 residual 数据 $\{\mathbf{x}_n, y_n - s_n\}$ 做回归,相当于每轮迭代逐步靠近标准答案 $y_n$。
所以 gradient boosting 中对 residual 做回归即得 $g_t$。

找到 $g_t$ 后,要找最优的 $\eta$:

$$\operatorname*{min}_{\eta} \frac{1}{N}\sum_{n=1}^{N} error \left ( s_n + \eta g_t(\mathbf{x}_n), y_n \right )$$

在平方错误下,即

$$\operatorname*{min}_{\eta} \frac{1}{N}\sum_{n=1}^{N} \left ( s_n + \eta g_t(\mathbf{x}_n) - y_n \right )^2 = \frac{1}{N}\sum_{n=1}^{N}((y_n - s_n) - \eta g_t(\mathbf{x}_n))^2$$

找最优 $\eta$ 相当于对用 $g_t$ 变换过的 $\mathbf{x}$ 和 residual 做单变量的线性回归,最优解为

$$\frac{\displaystyle \sum_{n=1}^{N}g_t(\mathbf{x}_n)(y_n - s_n)}{\displaystyle \sum_{n=1}^{N}(g_t(\mathbf{x}_n))^2}$$

\subsection{Gradient Boosted Decision Tree}
将上面 gradient boosting 结合决策树,就得到了 GBDT:

\begin{enumerate}
  \item 初始化所有 score $s_1 = s_2 = \dots = s_N = 0$ \\
  \item 对每轮迭代 $t=1,2,\dots,T$ \\
  \item 用决策树对 $\{\mathbf{x}_n, y_n - s_n\}$ 做回归,得到 $g_t$ \\
  \item 对 $\{g_t(\mathbf{x}_n), y_n - s_n\}$ 做单变量回归,得到 $\alpha_t$ \\
  \item 更新 $s_n \leftarrow s_n + \alpha_t g_t(\mathbf{x}_n)$
  \item 返回 $\displaystyle G(\mathbf{x}) = \sum_{t=1}^{T}\alpha_t g_t(\mathbf{x})$
\end{enumerate}

\subsection{Aggregation 的总结}

\begin{tabular}{p{2cm}|p{1cm}|p{3cm}|p{5cm}}
\hline
Blending & & Aggregation + Learning & \\
\hline
\hline
uniform      & 投票       & Bagging       & bootstrap + 投票 \\
\hline
             &            & Random Forest & 随机 bagging + DT \\
\hline
\hline
non-uniform  & 线性       & AdaBoost      & 调样本权重 + steepest descent \\
\hline
             &            & GradientBoost & 对 residual 回归 + steepest descent \\
\hline
             &            & GBDT          & GradientBoost + 弱 DT \\
\hline
\hline
conditional  & 非线性     & Decision Tree & 划分数据 + 分支函数实现条件 \\
\hline
\end{tabular}

\subsection{Neural Network}
基本思路是组合多个线性模型。以二元分类为例,假设有两组 $g$,分别对应
$\mathbf{w}_1$ 和 $\mathbf{w}_2$,如果给这俩的输出加上 $\alpha_1$ 和
$\alpha_2$ 的权值,则通过设置这些权值,可以实现布尔运算,从而形成非线性的分类边界。

常见的网络结构是,从输入 $\mathbf{x} = (x_1, x_2, \dots, x_d)$ 开始,
把所有特征依次和 $\mathbf{w}_1, \mathbf{w}_2, \dots, \mathbf{w}_j$
相乘得到 $j$ 个输出,然后输入经过 sigmoid 函数(常用 $\tanh$)得到下一层
的输入,依次经过所有 hidden layer 后,用最后一个 hidden layer 的 $\mathbf{w}$
组合得到输出。

符号 $w_{ij}^{(l)}$ 表示第 $l$ 层、第 $j$ 个节点对应的 $\mathbf{w}$ 的第 $i$ 个分量。
第 $l$ 层的节点总数为 $d^{(l)}$。第 $l$ 层、第 $j$ 个节点的输出为 $x_j^{(l)}$。

神经网络每一层都把前一层输出的 $\mathbf{x}$ 分别用一堆 $\mathbf{w}$ 相乘,
然后搞一下 sigmoid 函数,如果这个内积越大,说明 $\mathbf{x}$ 中这些特征
与 $\mathbf{w}$ 这种分配方式越接近,相当于用各层的 $\mathbf{w}$ 对原始的输入特征
进行变换,各层的 $\mathbf{w}$ 负责提取它对应的一些隐含特征。网络的学习
也就是要找到各层的 $\mathbf{w}$。

\subsection{NN 训练 \& 反向传播}
最后一层就是一个普通的线性模型(写出来就是一个求和式),以平方错误为例,
可以用随机梯度下降来优化 error 函数

$$e_n = (y_n - net(\mathbf{x}_n))^2$$

其中 $net(\cdot)$ 看做各层权重 $w_{ij}^{(l)}$ 的函数。

要求梯度就要对每个 $w_{ij}^{(l)}$ 求导嘛,但因为网络由多层,
需要用 chain rule,通过一系列中间变量来求。

最后一层(输出层):

$$e_n = (y_n - net(\mathbf{x}_n))^2 = (y_n - s_1^{(L)})^2 = \left ( y_n - \sum_{i=0}^{d^{(L-1)}} w_{i1}^{(L)}x_i^{(L-1)} \right )^2$$

求之

$$\frac{\partial e_n}{\partial w_{i1}^{(L)}} = \frac{\partial e_n}{\partial s_1^{(L)}} \cdot \frac{\partial s_1^{(L)}}{\partial w_{i1}^{(L)}} = -2(y_n - s_1^{(L)}) \cdot x_i^{(L-1)}$$

令

$$\delta_1^{(L)} = -2(y_n - s_1^{(L)})$$

即 $\delta_j^{(l)}$ 就是 $e_n$ 对第 $l$ 层第 $j$ 个节点 sigmoid 函数的输入 $s_j^{(l)}$ 求导的结果。

其他层可以递推,比如求导求到第 $l$ 层

$$\frac{\partial e_n}{\partial w_{ij}^{(l)}} = \frac{\partial e_n}{\partial s_j^{(l)}} \cdot \frac{\partial s_j^{(l)}}{\partial w_{ij}^{(l)}} = \delta_j^{(l)} \cdot x_i^{(l-1)}$$

现在要求 $\delta_j^{(l)}$,这个东西就是要找 $s_j^{(l)}$ 到 $e_n$ 的关系:


$$s_j^{(l)} \rightarrow x_j^{(l)} \rightarrow (s_1^{(l+1)}, \dots, s_k^{(l+1)}, \dots) \rightarrow \dots$$

所以

$$\delta_j^{(l)} = \frac{\partial e_n}{\partial s_j^{(l)}}
                 = \sum_{k=1}^{d^{(l+1)}} \frac{\partial e_n}{\partial s_k^{(l+1)}}
                                          \frac{\partial s_k^{(l+1)}}{\partial x_j^{(l)}}
                                          \frac{\partial x_j^{(l)}}{\partial s_j^{(l)}}
                 = \sum_{k} \delta_k^{(l+1)} w_{jk}^{(l+1)} (\tanh(s_j^{(l)}))^{\prime}$$

所以 $\delta$ 可以从后往前反向计算。

反向传播计算过程:
\begin{enumerate}
  \item 初始化各层 $w_{ij}^{(l)}$ \\
  \item 第 $t = 0, 1, \dots, T$ 轮迭代 \\
  \item 随机选择 $n \in \{1, 2, \dots, N\}$ \\
  \item 由 $\mathbf{x}^{(0)} = \mathbf{x}_n$ 正向计算出每层的输出 $x_i^{(l)}$ \\
  \item 由 $\mathbf{x}^{(0)} = \mathbf{x}_n$ 反向计算出每层 $\delta_j^{(l)}$ \\
  \item 更新所有 $w_{ij}^{(l)} \leftarrow w_{ij}^{(l)} - \eta x_i^{(l-1)} \delta_j^{(l)}$ \\
\end{enumerate}

最后返回 $\displaystyle g_{NN}(\mathbf{x}) = \left ( \dots \tanh \left ( \sum_{j}w_{jk}^{(2)} \cdot \tanh \left ( \sum_{i} w_{ij}^{(1)} x_i \right ) \right ) \right )$

实践上,一般每次随机多取几个点,然后并行执行前向和反向的计算,最后把各次计算出的 $x_i^{(l-1)} \delta_j^{(l)}$ 求平均后用于最后的更新中。
此法称为 mini-batch。

\subsection{NN 优化和 regularization}
前面说的用 SGD 训练出各个 $w$,但这个最优解可能只是局部最优,因为
整个网络的 error function 一般都不是凸函数。实践上的做法是在初始化时
设置较小的 $w$,或者尝试随机取 $w$。

NN 的 VC 维大小与神经元数量和 $w$ 数量有关,增加层数可能搞得很复杂很强大,但要小心 overfit。

Regularization 主要用 普通的 weight-decay regularizer

$$\sum(w_{ij})^2$$

或 weight-elimination regularizer

$$\sum\frac{(w_{ij}^{(l)})^2}{1 + (w_{ij}^{(l)})^2}$$

用这个东西的目的是使较大的 $w$ 减小,小的 $w$ 直接变为 0,让最后的 $w$ 比较稀疏。

另一种 regularization 的方法是及时停止梯度下降的迭代过程。
迭代次数越少可以看做是在 $w$ 变化的一个较小范围内做尝试,
及早停止直观上来看,有助于限制复杂度。

用 validation 选择不同的迭代次数。




\section{Week 7}
\subsection{Deep Learning Intro}
思想主要是用多层的网络,逐层提取特征,从比较 raw 的特征中逐步萃取
出高级特征然后做判断。对于 raw feature 较多的领域如语音、视觉有帮助。

主要难点
\begin{enumerate}
  \item 设计网络结构。可能需要特定领域的知识\\
  \item 控制模型复杂度。设计 regularization 方法\\
  \item 优化目标函数。非 convex,容易陷入 local optimum,应合理设初值\\
  \item 加速计算。一般都往 GPU 搞\\
\end{enumerate}

\subsection{Pre-train \& Autoencoder}
Pre-train 即一种设初值的方法,通过简单的 2 层网络来学习初值。

初值也就是权重 $w_{ij}^{(l)}$ 的初值,权重做的就是特征转换,为有助于后面学习,
需要尽可能地保持原始输入的信息,并根据每层的网络结构,将前层输入转换为
后层的输入,形成 $d-\tilde{d}-d$ 网络(通常 $\tilde{d} < d$ 即做压缩)
目标是让最后的输出 $g_i(\mathbf{x}) \approx x_i$,
即所谓 autoencoder,前后两层权重分别称 encoding/decoding weights。
Autoencoder 做的即是学出原始数据的一种呈现方式。

Autoencoder 的训练所用 error function 为

$$\sum_{i=1}^{d}(g_i(\mathbf{x}) - x_i)^2$$

训练数据即 $\{(\mathbf{x}_n, y_n = \mathbf{x}_n)\}$,通常当做 unsupervised。

简单的 regularization 方法是令网络前后层 $w_{ij}^{(1)} = w_{ji}^{(2)}$。

一般每两层之间学一个 autoencoder。

作为一种降低噪音影响的 regularization 方法,常在训练 autoencoder 时
在 target 中加入人为的噪音,即训练数据变为 $\{(\tilde{\mathbf{x}}_n, y_1=\mathbf{x}_n)\}$,
其中 $\tilde{\mathbf{x}}_n = \mathbf{x}_n + noise$。此法相当于
提示 autoencoder 要将噪音剔除,找出数据的真相。


\subsection{Autoencoder \& PCA}
考虑在 autoencoder 中做几个改变
\begin{enumerate}
  \item 不考虑 $x_0$ 输入\\
  \item 前后权重相等 $w_{ij}^{(1)} = w_{ji}^{(2)} = w_{ij}$\\
  \item $\tilde{d} < d$\\
\end{enumerate}

则 autoencoder 输出的第 $k$ 个分量为

$$h_k(\mathbf{x}) = \sum_{j=1}^{\tilde{d}}w_{jk}^{(2)}\left ( \sum_{i=1}^{d}w_{ij}^{(1)}x_i \right )$$

根据以上条件可以改写为

$$h_k(\mathbf{x}) = \sum_{j=1}^{\tilde{d}}w_{kj}\left ( \sum_{i=1}^{d}w_{ij}x_i \right )$$

令矩阵 $\mathbf{W} = (w_{ij})_{d \times \tilde{d}}$,autoencoder 的输出向量为

$$\tilde{\mathbf{x}} = h(\mathbf{x}) = \mathbf{WW}^{T}\mathbf{x}$$

训练 autoencoder 要优化的目标函数


$$E_{in}(h) = E_{in}(\mathbf{W}) = \frac{1}{N}\sum_{n=1}^{N}\| \mathbf{x}_n - \mathbf{WW}^{T}\mathbf{x}_n \|^2$$

因为 $\mathbf{WW}^{T}$ 对称,所以可以做特征值分解为 $\mathbf{VDV}^{T}$,
且它最多有 $\tilde{d}$ 个非零特征值(因为特征值最多 rank 个,而 $\mathbf{WW}^T$ 半正定且和 $\mathbf{W}$ 等 rank)。

所以上面的 $E_{in}$ 可以写为

$$\operatorname*{min}_{\mathbf{V}} \operatorname*{min}_{\mathbf{D}} \frac{1}{N}\sum_{i=1}^{N}\left\|\mathbf{VIV}^T\mathbf{x}_n - \mathbf{VDV}^T\mathbf{x}_n \right\|$$

根据正交对角化的几何意义,$\mathbf{VDV}^T$ 作用在一个向量上,
相当于将它做坐标变换,然后扔掉至少 $d - \tilde{d}$ 个分量,再把其他分量做 scaling,
最后把坐标转换回去。

同时,$\mathbf{V}$ 正交,保长度,所以不影响优化,内层的优化相当于

$$\operatorname*{min}_{\mathbf{D}} \sum \|(\mathbf{I-D})\cdot\dots\|^2$$

显然要让它最大,也就是让 $\mathbf{I-D}$ 中有尽可能多的 0 元素,
也就是说要

$$
\operatorname*{min}_{\mathbf{V}}\sum_{n=1}^{N}\left\|\begin{bmatrix}
0 & 0 \\
0 & \mathbf{I}_{d-\tilde{d}} \\
\end{bmatrix} \mathbf{V}^T\mathbf{x}_n\right\|^2
$$

也就是

$$
\operatorname*{max}_{\mathbf{V}}\sum_{n=1}^{N}\left\|\begin{bmatrix}
\mathbf{I}_{\tilde{d}} & 0 \\
0 & 0 \\
\end{bmatrix} \mathbf{V}^T\mathbf{x}_n\right\|^2
$$

如果 $\tilde{d}=1$,那就只有 $\mathbf{V}^T$ 的第一行 $\mathbf{v}$ 会起作用,
相当于

$$\operatorname*{max}_{\mathbf{v}}\sum_{n=1}^{N}\mathbf{v}^T\mathbf{x}_n\mathbf{x}_n^{T}\mathbf{v}$$

约束为 $\mathbf{v}^T\mathbf{v}=1$,因为 $\mathbf{V}$ 里面都是规范正交化的向量。

这个东西根据拉格朗日乘数法,可知最优的 $\mathbf{v}$ 应满足

$$\sum_{n=1}^{N}\mathbf{x}_n\mathbf{x}_n^T\mathbf{v}=\lambda \mathbf{v}$$

也就是说最优解必须是 $\mathbf{X}^T\mathbf{X}$ 的特征向量。这个东西
代入原来要优化的式子可得最大值为 $\lambda$,所以我们要找的最优的 $\mathbf{v}$
就等于$\mathbf{X}^T\mathbf{X}$ 的最大特征值对应的特征向量。

同理,对一般的 $\tilde{d}$,可以得到类似结论,即最优解必须是$\mathbf{X}^T\mathbf{X}$ 的最大的几个特征值对应的特征向量。

直观上来看,autoencoder 要做的是找到与 $\mathbf{x}_n$ 最 “match” 的权重。

这个东西基本就是 PCA。也可以在做 PCA 之前先把每个数据变为

$$\mathbf{x}_n \leftarrow \mathbf{x}_n - \bar{\mathbf{x}}$$

即将均值归为 0,然后计算 $\mathbf{X}^T\mathbf{X}$ 最大的 $\tilde{d}$ 个特征向量,
最后返回 encode 后的特征 $\Phi(\mathbf{x}) = \mathbf{W}(\mathbf{x} - \bar{\mathbf{x}})$。

Autoencoder 也可以看做是降维,找到尽量保持原数据信息的表示方式,并降低 $d$ 到 $\tilde{d}$。

\subsection{Radial Basis Function Network}
RBF 就是一种距离衡量指标,RBF 网络就是用一堆的中心来对数据进行类似投票
的动作,然后把所有结果线性组合起来,即

$$h(\mathbf{x}) = output \left ( \sum_{m=1}^{M}\beta_m RBF(\mathbf{x},\bm{\mu}_m)+b \right )$$

其中 $\bm{\mu}_m$ 即各个 RBF 的中心,$\beta_m$ 为投票权重。

在 Gaussian RBF + SVM 模型中,$M$ 即支持向量数,$\bm{\mu}_m$ 为支持向量,
$\beta_m$ 即 SVM dual 的 $\alpha_m y_m$。

RBF 网络学习任务即找到 $\bm{\mu}_m$ 和 $\beta_m$。

\subsection{RBF Network Training \& k-Means}
若中心数 $M = N$,称 full RBF network,每个中心值 $\bm{\mu}_m = \mathbf{x}_m$。
此举之思想即认为每个 $\mathbf{x}_m$ 都会影响与其相似的 $\mathbf{x}$,
例如对于二元分类,每个 $\mathbf{x}_m$ 按 $\mathbf{x}$ 与之距离为其投 $y_m$。

因为 RBF 是距离指标,所以可以简化为只考虑与 $\mathbf{x}$ 最近的一个或几个 $\mathbf{x}_m$,
称为 k-nearest neighbor (KNN)。此法与 full RBF network 一样,都是在训练时偷懒,
测试时要花费较高的计算代价。

RBF network 做的事情相当于对每个 $\mathbf{x}_n$ 做特征变换为

$$\mathbf{z}_n = (RBF(\mathbf{x}_n, \mathbf{x}_1), RBF(\mathbf{x}_n, \mathbf{x}_2), \dots, RBF(\mathbf{x}_n, \mathbf{x}_N))$$

然后把这些 $\mathbf{z}_n$ 列为$N\times N$ 对称矩阵 $\mathbf{Z}$,类似于 kernel 里的 matrix。

对 Gaussian RBF,只要每个 $\mathbf{x}_n$ 都不同,得到的矩阵 $\mathbf{Z}$ 便可逆。

可以计算,full RBF network + 线性回归会得到 $E_{in}=0$,通常需要 regularization。

注意 full RBF network + regularization 和 kernel ridge regression + regression 得到的结果不同,
因为 full RBF network 是在原空间做 regularization,而后者是在 kernel 的空间中做 regularization。

另一种 regularization 的方法是限制中心数量 $M \ll N$,那么学习任务就是要找到合适的代表来作为中心。
注意若 $\mathbf{x}_1 \approx \mathbf{x}_2$,那么他们可以用同一个中心来代表。所以找中心的过程
即 clustering,将所有数据分为 $M$ 个集合,每个集合有自己的中心 $\bm{\mu}_m$,使得
整体 clustering 的 error 最小:

$$\operatorname*{min}_{\{S_1,\dots,S_M;\bm{\mu}_1,\dots,\bm{\mu}_M\}} E_{in}(S_1, \dots, S_M; \bm{\mu}_1, \dots, \bm{\mu}_M) = \frac{1}{N}\sum_{n=1}^{N}\sum_{m=1}^{M}\mathbf{1}\{\mathbf{x}_n \in S_m\}\|\mathbf{x}_n-\bm{\mu}_m\|^2$$

这个优化很难直接做,所以先考虑当 $\bm{\mu}_m$ 都固定时,要做分组的动作,
只需要找到距离每个 $\mathbf{x}_n$ 最近的 $\bm{\mu}_m$ 即可。

而当集合固定时,要找最优的中心,可以对上式所有 $\bm{\mu}_m$ 求导得

$$\nabla E_{in} = -2\sum_{n=1}^{N}\mathbf{1}\{\mathbf{x}_n \in S_m\}(\mathbf{x}_n - \bm{\mu}_m) = -2\left ( \left ( \sum_{\mathbf{x}_n \in S_m}\mathbf{x}_n  - |S_m|\bm{\mu}_m\right ) \right )$$

显然这个最优解应该是取每个 $S_m$ 的均值。

于是根据上面这种交替优化的步骤得到 k-means:

\begin{enumerate}
  \item 初始化 $\bm{\mu}_k$ 为 $k$ 个随机选择的 $\mathbf{x}_n$ \\
  \item 重复交替优化步骤:先按 $\bm{\mu}$ 分组,再求各组均值更新 $\bm{\mu}$\\
  \item 不断迭代直到收敛\\
\end{enumerate}

收敛性是可以保证的:因为优化过程不断减小 $E_{in}$,而其下限为 0。
k-means 对中心的初始化比较敏感。

将 k-means 结合 RBF network 就得到 RBF network 的训练过程:

\begin{enumerate}
  \item 用 k-means 找到 $M$ 个中心\\
  \item 对数据做转换 $\Phi(\mathbf{x})=(RBF(\mathbf{x},\bm{\mu}_1),RBF(\mathbf{x},\bm{\mu}_2),\dots,RBF(\mathbf{x},\bm{\mu}_M))$\\
  \item 在 $\{(\Phi(\mathbf{x}_n), y_n)\}$ 上用线性模型找到投票权重 $\beta$\\
  \item 返回最终 hypothesis\\
\end{enumerate}


\section{Week 8}
\subsection{Matrix Factorization Intro}
问题:电影推荐,数据是 abstract feature,即非具有特定含义的数值。
总共 $N$ 个用户,$M$ 部电影,
数据记录了某用户 $n$ 为某电影 $m$ 的评分 $r_{nm}$。对第 $m$ 部
电影而言,与之相关的所有用户评分数据为

$$\mathcal{D}_m = \{(\tilde{\mathbf{x}_n}=(n), y_n=r_{nm})\}$$

现在把这种 categorical feature 改写为 binary encoded vector,即
每种 category 都在自己对应的位置上有 1,否则为 0,上面的数据变为

$$\mathcal{D}_m = \{(\mathbf{x}_n=[0, 0, \dots, 1, \dots, 0]^T, y_n=r_{nm})\}$$

再把每个用户的所有评分都聚合起来

$$\mathcal{D}_m = \{(\mathbf{x}_n=[0, 0, \dots, 1, \dots, 0]^T, \mathbf{y}_n=[r_{n1}, \dots, r_{nM}]^T)\}$$

注意其中有些 $r_nm$ 是未知值,因为此用户可能没看过某些电影。

现在的任务时,从这种抽象特征中提取可用的特征。采用类似 autoencoder 的两层
网络来实现,网络中不含有每层的 $x_0$ 输入,也不用 sigmoid 函数。

将第一层的所有权重记为矩阵 $\mathbf{V}_{\tilde{d}\times N}$,第二层权重记为矩阵
$\mathbf{W}_{\tilde{d}\times M}$,所以整个网络的输出为

$$h(\mathbf{x})=\mathbf{W}^T\mathbf{Vx}$$

因为输入 $\mathbf{x}_n$ 只有第 $n$ 位为 1,其他都为 0,所以 $\mathbf{Vx}$ 相当于
只提取了矩阵 $\mathbf{V}$ 的第 $n$ 列,即

$$h(\mathbf{x}_n)=\mathbf{W}^T\mathbf{v}_n$$

进一步,把 $\mathbf{Vx}$ 看做特征转换,对于第 $m$ 部电影,只有 $\mathbf{W}^T$ 的
第 $m$ 行参与计算,所以对于第 $m$ 部电影

$$h_m(\mathbf{x})=\mathbf{w}_m^T\Phi(x)$$

也就是对每个电影都是一个线性模型。

对全部数据而言,根据平方错误,有

$$E_{in}(\mathbf{w}_m, \mathbf{v}_n) = \frac{1}{\displaystyle\sum_{m=1}^M|\mathcal{D}_m|}\sum_{n, m}(r_{nm} - \mathbf{w}_m^T\mathbf{v}_n)^2$$

现在把每个用户的所有评分写成矩阵 $\mathbf{R}_{N\times M}$,则我们的任务是让

$$\mathbf{w}_m^T\mathbf{v}_n \approx r_{nm}$$

即

$$\mathbf{R} \approx \mathbf{V}^T\mathbf{W}$$

也就是说,要找到一个矩阵分解。

\subsection{Matrix Factorization Learning}
根据 loss

$$\operatorname*{min}_{\mathbf{W, V}} E_{in}(\mathbf{w}_m, \mathbf{v}_n) \propto \sum_{n, m}(r_{nm} - \mathbf{w}_m^T\mathbf{v}_n)^2 = \displaystyle\sum_{m=1}^{M}\left (\displaystyle \sum_{(\mathbf{x}_n,r_{nm}) \in \mathcal{D}_m}(r_{nm}-\mathbf{w}_m^T\mathbf{v}_n) \right )^2$$

此式用类似于 k-means 的方法做交替优化,因为固定 $\mathbf{v}_n$ 时,
每个 $\mathbf{w}_m$ 只在对应的项出现在 sum 中才会影响求导。而内积是可以交换的,
所以 $\mathbf{w}_m$ 和 $\mathbf{v}_n$ 的位置是同等的,得到交替优化的算法:

\begin{enumerate}
  \item 随机初始化 $\mathbf{w}_m$, $\mathbf{v}_n$ \\
  \item 优化 $\mathbf{w}_m$,对 $\{(\mathbf{v}_n, r_{nm})\}$ 做线性回归\\
  \item 优化 $\mathbf{v}_n$,对 $\{(\mathbf{w}_m, r_{nm})\}$ 做线性回归\\
  \item 直到收敛\\
\end{enumerate}

此法史称 alternating least squares。

\subsection{Learn by SGD}
也可以用 SGD 来优化,因为内层求梯度之后外面还有个求和,符合 SGD 的模式。
Error function

$$error(n, m, r_{nm}) = (r_{nm} - \mathbf{w}_m^T\mathbf{v}_n)^2$$

梯度

$$\nabla_{\mathbf{v}_n}error = -2(r_{nm}-\mathbf{w}_m^T\mathbf{v}_n)\mathbf{w}_m$$
$$\nabla_{\mathbf{w}_m}error = -2(r_{nm}-\mathbf{w}_m^T\mathbf{v}_n)\mathbf{v}_n$$

优化算法:

\begin{enumerate}
  \item 随机初始化 $\mathbf{w}_m$, $\mathbf{v}_n$ \\
  \item 每一轮 $t = 0, 1, \dots, T$\\
  \item 随机选择 $r_{nm}$\\
  \item 计算 $\tilde{r}_{nm} = r_{nm} - \mathbf{w}_m^T\mathbf{v}_n$\\
  \item SGD 更新 $\mathbf{v}_n^{new} \leftarrow \mathbf{v}_n^{old} + \eta \cdot \tilde{r}_{nm}\mathbf{w}_{m}^{old}$,$\mathbf{w}_m^{new} \leftarrow \mathbf{w}_m^{old} + \eta \cdot \tilde{r}_{nm}\mathbf{v}_{n}^{old}$
\end{enumerate}


\end{document}

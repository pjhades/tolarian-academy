\documentclass[a4paper]{article}
\usepackage{xeCJK}
\usepackage{amsmath}
\usepackage{amsfonts}
\usepackage{amssymb}
\usepackage{color}
\usepackage[colorlinks=true]{hyperref}
\setCJKmainfont{STKaiti}
\setCJKsansfont{STKaiti}
\setCJKmonofont{Monaco}

\begin{document}
\title{Coursera 機器學習基石}
\maketitle

\section{Week 1}
\subsection{Perceptron Learning Algorithm (PLA)}
简单的线性分类模型。输入 $\mathcal{D} = \{(\mathbf{x}_1, y_1), (\mathbf{x}_2, y_2), \dots, (\mathbf{x}_N, y_N)\}$,
每个 $\mathbf{x_i} = (x_1, x_2, \dots, x_d)$ 维度为 $d$,对应 $d$ 种特征。输出即类别 $\mathcal{Y} = \{-1, +1\}$。

考虑对每个特征赋一个权值 $w_i$,所有特征的权值构成权值向量 $\mathbf{w}$,对每个输入可以计算出总的分数,
并根据分数是否超过某阈值 $threshold$ 来决定判定输出:

$$ h(\mathbf{x}) = sign(\displaystyle\sum_{i=1}^{d} w_ix_i - threshold) $$

令 $w_0 = -threshold$,$x_0 = 1$,上式可简化为:

$$ h(\mathbf{x}) = sign(\displaystyle\sum_{i=0}^{d} w_ix_i) = sign(\mathbf{w}^T \cdot \mathbf{x}) $$

算法开始时初始化 $\mathbf{w} = \mathbf{w_0}$,然后基于训练数据不断对 $\mathbf{w}$ 做出修正,直到 $\mathbf{w}$ 将训练数据
全部正确分类。

若在第 $t$ 轮迭代中,PLA 发现对训练数据 $\mathbf{x}_{n(t)}$ 分类错误,即:

$$sign(\mathbf{w}_t^T \cdot \mathbf{x}_{n(t)}) \neq y_{n(t)}$$

则可以做出修正:

$$\mathbf{w}_{t+1} \leftarrow \mathbf{w}_t + y_{n(t)}\mathbf{x}_{n(t)}$$

因为内积运算结果的符号由向量夹角的余弦表示,所以若发现分类结果与“标准答案”异号,则可以通过以上修正让 $\mathbf{w}_{t+1}$
更加接近 $\mathbf{x}_{n(t)}$。

如此不断更新 $\mathbf{w}_t$ 直到其对所有训练数据都能正确分类。训练数据可以随机顺序访问,或按照某个预先设定的顺序循环访问。

\subsection{PLA 的收敛性}
为什么 PLA 不会一直运行下去?

设 $\mathbf{w}_f$ 为我们要学习的目标权重向量(未知)。如果数据线性可分,则必有:

$$\min_{n}y_n\mathbf{w}_f^T\mathbf{x} > 0$$

即,从训练数据中计算出来的最小总分和“参考答案”也必须是同号的。

同时最优的那个 $\mathbf{w}_f$ 要满足:

$$y_{n(t)}\mathbf{w}_f^T\mathbf{x}_{n(t)} \ge \min_{n}y_n\mathbf{w}_f^T\mathbf{x} > 0$$

联想一下点到直线的距离公式,最优的 $\mathbf{w}_f$ 到各个数据点应该具有一个最小的 margin。

首先证明 PLA 更新之后内积会逐渐变小,方法是比较更新后的权重向量和最优向量:

\begin{equation}
\label{pla-decrease}
\begin{aligned}
\mathbf{w}_f^T\mathbf{w}_{t+1} &= \mathbf{w}_f^T(\mathbf{w}_t + y_{n(t)}\mathbf{x}_{n(t)}) \\
                               &= \mathbf{w}_f^T\mathbf{w}_t + y_{n(t)}\mathbf{x}_f^T\mathbf{x}_{n(t)} \\
                               &\ge \mathbf{w}_f^T\mathbf{w}_t + \min_{n}y_n\mathbf{w}_f^T\mathbf{x}_n \\
                               &> \mathbf{w}_f^T\mathbf{w}
\end{aligned}
\end{equation}

接下来证明 PLA 每次更新的幅度不大,方法是估计更新后 $\mathbf{w}_{t+1}$ 的范数的上界。

注意到仅有出错时才更新。出错时 $y_{n(t)}\mathbf{w}_t^T\mathbf{x}_{n(t)} \le 0$,即异号,所以有:

\begin{equation}
\label{not-so-large}
\begin{aligned}
\|\mathbf{w}_{t+1}\|^2 &= \|\mathbf{w}_t + y_{n(t)}\mathbf{x}_{n(t)}\|^2 \\
                       &= \|\mathbf{w}_t\|^2 + 2y_{n(t)}\mathbf{w}_t^T\mathbf{x}_{n(t)} + \|y_{n(t)}\mathbf{x}_{n(t)}\|^2 \\
                       &\le \|\mathbf{w}_t\|^2 + \|y_{n(t)}\mathbf{x}_{n(t)}\|^2 \\
                       &\le \|\mathbf{w}_t\|^2 + \max_{n}\|y_n\mathbf{x}_n\|^2
\end{aligned}
\end{equation}

所以更新之后 $\|\mathbf{w}_{t+1}\|^2$ 最多比 $\|\mathbf{w}_t\|^2$ 增长 $\displaystyle\max_{n}\|y_n\mathbf{x}_n\|^2$ (最远的点)。

设 PLA 最多进行 $T$ 次更新,由 \ref{pla-decrease} 得:

$$\frac{\mathbf{w}_f^T\mathbf{w}_T}{\|\mathbf{w}_f\|} \ge T \cdot \min_{n}\frac{y_n\mathbf{w}_f^T\mathbf{x}_n}{\|\mathbf{w}_f\|} = T\rho$$

由 \ref{not-so-large} 得:

$$\|\mathbf{w}_T\|^2 \le TR^2$$

综上,

$$1 \ge \frac{\mathbf{w}_f^T\mathbf{w}_T}{\|\mathbf{w}_f\| \cdot \|\mathbf{w}_T\|} \ge \frac{T\rho}{\sqrt{T}R} = \sqrt{T} \cdot \frac{\rho}{R}$$

可得更新次数的上界为

$$T \le \frac{R^2}{\rho^2}$$


\subsection{非线性可分问题}
Pocket 在 PLA 基础上,维护几个量:

\begin{enumerate}
  \item 当前最佳 $\mathbf{w}_t$
  \item 当前连续正确分类的次数
  \item 当前最佳连续分类次数
\end{enumerate}

每次正确分类后,若连续正确分类次数超过历史最优,则计算当前 $\mathbf{w}_t$ 对所有训练数据的分类错误率,若犯错比历史值更少,
则将当前 $\mathbf{w}_t$ 保存为 $\hat{\mathbf{w}}$。超过预设更新次数后,输出 $\hat{\mathbf{w}}$。





\section{Week 2}
\subsection{Learning 是否可行?}
实际的 hypothesis $f$ 是永远无法获知的,无论如何设计学习算法,都有一些“个人考虑”在里面,
不能达到最终的 $f$,例子:$3 \times 3$ 黑白方格分类问题。所以从 possibility 的角度讲,
learning 是 impossible 的。但 impossible 并不代表 not probable。所以需要建立关于“learning is probable”的理论基础。

所谓算法“学到东西”,是指学习算法输出的 $g$ 在测试数据上也要有较好的表现。
那么,算法在有限的训练数据上的表现,能否提供训练数据之外的信息呢?

\subsection{Bin model 和 Hoeffding 不等式}
考虑装有红绿两色球的罐子,要估计罐子中红球的比例 $\mu$,独立地随机抽 $N$ 个球,通过样本中红球的比例 $\nu$ 来估计 $\mu$。
此模型满足 Hoeffding 不等式

$$\mathbb{P}[|\mu - \nu| > \epsilon] \le 2e^{-2N\epsilon^2}$$

其实就是大数定理的一个形式,要满足较小的误差,就需要抽足够大的样本。Hoeffding 不等式即给出了“误差太大”这一坏事发生的概率上界。

将 bin model 联系到 learning 当中,对于任意一个确定的 $h$,对训练数据 ${x_1, x_2, \dots, x_N}$ 依次计算 $h(x_n)$,$n = 1, 2, \dots, x_N$,若 $h(x_n) \neq y_n$ 则将球涂成红色(出错),否则涂绿色,即可得到一个样本。如果训练数据和测试数据都由同一个分布 $P(\mathcal{X})$ 生成,即可根据 Hoeffding 不等式,用算法在训练数据上的错误率 $\nu$ 来估计其在测试数据上的错误率 $\mu$。
此时 $\nu$ 和 $\mu$ 分别称为 in-sample error 和 out-of-sample error,分别记为 $E_{in}$ 和 $E_{out}$,即

\begin{equation}
\label{hoeffding-ein-eout}
\mathbb{P}[|E_{in} - E_{out}| > \epsilon] \le 2e^{-2N\epsilon^2}
\end{equation}

注意到这里应用 Hoeffding 不等式的前提是固定 $h$,所以不等式实际上是为 verification 提供了理论保证:对某个 $h$,$E_{in}$ 和 $E_{out}$ 差别不会太大。

但学习算法是在 $\mathcal{H}$ 中选定一个 $g$,所以我们希望把 Hoeffding 不等式提供的保障推广到 $|\mathcal{H}| = M$ 个 hypothesis
的情况。取 union bound,只要 $\mathcal{H}$ 中任意一个 $h$ 可能发生“坏事”,则最后选出来的 $g$ 也可能发生,即

\begin{align*}
\mathbb{P}[|E_{in}(g) - E_{out}(g)| > \epsilon] &= \mathbb{P}[\bigvee_{1 \le m \le M}\left(E_{in}(h_m) - E_{out}(h_m) > \epsilon \right)] \\
												&\le \sum_{1 \le m \le M}\mathbb{P}[|E_{in}(h_m) - E_{out}(h_m)|] \\
												&=   \sum_{1 \le m \le M}2e^{-2N\epsilon^2} \\
												&= 2Me^{-2N\epsilon^2}
\end{align*}

这个上界的特点:
\begin{enumerate}
  \item 非常松
  \item 依赖于模型的复杂程度 $M$:若要 $E_{in}(g)$ 接近 $E_{out}(g)$,需要 $M$ 小,但太简单的模型往往不理想,即 $E_{in}(g)$ 会比较大
  \item 对于 $M = +\infty$ 没有意义,但很多问题都有 $M = +\infty$,PLA 就是
\end{enumerate}


\section{Week 3}
\subsection{Error bar}
现在我们知道

$$\mathbb{P}[|E_{in}(g) - E_{out}(g)| > \epsilon] \le 2Me^{-2N\epsilon^2}$$

假设允许坏事发生的概率最大为 $\delta$,即至少可以 $1 - \delta$ 的概率保证 $|E_{in}(g) - E_{out}(g)| \le \epsilon$,
变形得:

$$E_{out}(g) \le E_{in}(g) + \sqrt{\frac{1}{2N}\ln\frac{2M}{\delta}}$$

右边根号项称为 error bar,即将 $g$ 从训练数据推广到测试数据产生的误差的界。因为带有 $M$,所以不能处理 $M = +\infty$ 的情况。

\subsection{寻找更紧的上界}
主要想法是缩小 $\mathcal{H}$ 的大小。以 PLA 为例,两条分类线如果斜率很接近,分类结果也会很接近,$E_{in}$ 也会很接近。
尝试找出 $\mathcal{H}$ 的“有效大小”,实际在 $M = +\infty$ 时也是有限值,从而形成一个上界。

\paragraph{Dichotomy}
定义 $\mathcal{H}$ 在 ${\mathbf{x}_1, \mathbf{x}_2, \dots, \mathbf{x}_N} \in \mathcal{X}$
上形成的 dichotomy 为:

$$\mathcal{H}(\mathbf{x}_1, \mathbf{x}_2, \dots, \mathbf{x}_N) = \{ \, (h(\mathbf{x}_1), h(\mathbf{x}_2), \dots, h(\mathbf{x}_N)) \, | \, h \in \mathcal{H} \, \}$$

一个 dichotomy 即所有 $h$ 在一些数据上可能形成的结果,表明 $\mathcal{H}$ 的“多样性”。

\paragraph{Shattered set}
设 class $C$ 和集合 $A$,如果对每个 $T \subset A$ 都存在 $U \subset C$ 使得
$U \cap A = T$,则称 $C$ \textit{shatter} $A$。即,当幂集 $P(A) = \{U \cap A \, | \, U \in C\}$ 时,
$C$ shatters $A$。

例如,二维 PLA 无法 shatter 平面上任意 4 个点,但可以 shatter 任意 3 个点。

\paragraph{Growth function}
为了衡量一个 $\mathcal{H}$ shatter 数据集能力的能力,定义 $\mathcal{H}$ 的 growth function 为:

$$m_{\mathcal{H}}(N) = \max_{\mathbf{x}_1, \mathbf{x}_2, \dots, \mathbf{x}_N \in \mathcal{X}}|\mathcal{H}(\mathbf{x}_1, \mathbf{x}_2, \dots, \mathbf{x}_N)|$$

即这个 $\mathcal{H}$ 在任意 $N$ 个点上能产生的最大 dichotomy 大小。对于二元分类问题,显然 $m_{\mathcal{H}}(N) \le 2^N$。

$m_{\mathcal{H}}(N)$ 满足性质:
\begin{enumerate}
  \item 若 $m_{\mathcal{H}}(N) = 2^N$,则表明存在大小为 $N$ 的集合,能够被 $\mathcal{H}$ shatter,因为从中可以拿出集合去求交,得到幂集中所有可能的集合大小
  \item 若对某个 $n > 1$ 有 $m_{\mathcal{H}}(m) < 2^m$,那么对所有 $n > m$ 也有 $m_{\mathcal{H}}(n) < 2^n$,因为如果 $\mathcal{H}$ 不能 shatter 某个大小 $m$,更大的集合中仍然有不能 shatter 的情况存在
\end{enumerate}

\paragraph{Break point}
若 $\mathcal{H}$ 不能 shatter 大小为 $k$ 的数据集,则称 $k$ 是 $\mathcal{H}$ 的一个 break point。
由此可得,若 $\mathcal{H}$ 有 break point $k$,则 $m_{\mathcal{H}}(k) < 2^k$。

\paragraph{Growth function 的界}
在 break point 存在的情况下,hypothesis 的有效数量会被极大地限制,所以要找到 growth function 的上界。
而如果我们能用 growth function 取代 error bar 中的 $M$,那么,只要 growth function 有多项式的上界,
error bar 右边取对数之后,不管多项式次数多少,都是按 $N$ 对数增长,当 $N$ 足够大的时候,$E_{in}$ 和 $E_{out}$
就有可能非常接近。

设 $B(N, k)$ 为 $N$ 个点中任意 $k$ 个都不被 shatter 的最大 dichotomy 数量。如果 $m_{\mathcal{H}}(N)$ 有 break point $k$,则

$$m_{\mathcal{H}}(N) \le B(N, k)$$

因为 growth function 衡量的是 $\mathcal{H}$ 的“划分”能力,而 $B(N, k)$ 是任意 $N$ 个点的最大 dichotomy 数。

$B(N, k)$ 显然满足

\begin{equation}
\begin{aligned}
B(N, 1) &= 1 \\
B(1, k) &= 2 \qquad for \qquad k > 1 \\
\end{aligned}
\end{equation}

现在设 $N \ge 2$、$k \ge 2$,尝试找到 $B(N, k)$ 的递推关系。假设有 $B(N, k)$ 个 dichotomy,我们把这些 dichotomy 分为两类:
\begin{enumerate}
  \item $\mathcal{H}_1 = \{ \,  (h(\mathbf{x}_1), h(\mathbf{x}_2), \dots, h(\mathbf{x}_{N-1})) \, \}$,即 $\mathcal{H}$ 在前 $N-1$ 个点上产生的 dichotomy  
  \item $\mathcal{H}_2 = \{ \,  (y_1, y_2, \dots, y_{N-1}) \in \mathcal{H}_1 \, | \, \exists h, h^{\prime} \in \mathcal{H} \, s.t. \, h(\mathbf{x}_i) = h^{\prime}(\mathbf{x}_i) = y_i, \, \forall i \in [1, N-1] \wedge h(\mathbf{x}_N) \neq h^{\prime}(\mathbf{x}_N) \, \}$,即仅对 $\mathbf{x}_N$ 结果不同的所有 $h$ 在前 $N-1$ 个点上产生的 dichotomy
\end{enumerate}

则有

$$B(N, k) = |\mathcal{H}_1| + |\mathcal{H}_2|$$

由以上定义,设仅在 $\mathbf{x}_N$ 上不同的 hypothesis 产生的 dichotomy 数量为 $2\beta$,剩下 dichotomy 数量为 $\alpha$,则

$$B(N, k) = \alpha + 2\beta$$

这 $\beta$ 个 dichotomy 显然满足

$$\beta \le B(N-1, k-1)$$

否则若加上 $\mathbf{x}_N$ 上的结果,就会 shatter $k+1$ 个点,不满足 $B(N, k)$ 的假设。

另外,因为 $B(N, k)$ 保证了没有任意 $k$ 个点被 shatter,所以不考虑 $\mathbf{x}_N$,就应该有

$$\alpha + \beta \le B(N-1, k)$$

以上两不等式相加,得:

$$B(N, k) = \alpha + \beta \le B(N-1, k-1) + B(N-1, k)$$

在此基础上,有 Sauer 引理:

$$B(N, k) \le \sum_{i=0}^{k-1}\binom{N}{i}$$

根据 $B(N, k)$ 的递推关系用数学归纳法即可证明,或参见\href{http://www.shivani-agarwal.net/Teaching/E0370/Aug-2011/Lectures/4.pdf}{这里}。

由此可以得出 growth function 的上界

$$m_{\mathcal{H}}(N) \le B(N, k) \le \sum_{i=0}^{k-1}\binom{N}{i}$$



\section{Week 4}

\subsection{VC Dimension}
上面那个 $k-1$ 即 $\mathcal{H}$ 的 VC 维,记为 $d_{VC}(\mathcal{H})$,也就是 $\mathcal{H}$ 能够 shatter 的最大点数。
若对所有 $N$ 都有 $m_{\mathcal{H}}(N) = 2^N$,则 $d_{VC}(\mathcal{H}) = \infty$。

\paragraph{VC Generalization Bound}
对任意 $\delta > 0$,下式以概率 $\ge 1 - \delta$ 成立:

$$E_{out}(g) \le E_{in}(g) + \sqrt{\frac{8}{N}\ln{\frac{4m_{\mathcal{H}}(2N)}{\delta}}}$$

即 Vapnik-Chervonenkis 不等式

$$\mathbb{P} \left [ \sup_{h \in \mathcal{H}}|E_{in}(h) - E_{out}(h)| > \epsilon \right ] \le 4m_{\mathcal{H}}(2N)e^{-\frac{1}{8}N\epsilon^2}$$

注意几点:
\begin{enumerate}
  \item $m_{\mathcal{H}}(N)$ 替代了原来的 $M$。 因为对于某个 $\mathcal{D}$,很多 hypothesis 会共享一些 dichotomy,
  所以在计算 error 时,这些 hypothesis 要错一起错。Growth function 能做的就是计算这种重复。书里举了个例子,因为训练数据 $\mathcal{D}$
  是随机性的唯一来源,想象所有 $\mathcal{D}$ 的空间为一张白纸,某个 $\mathcal{D}$ 如果最后 $E_{in}(g)$ 和 $E_{out}(g)$ 相差很大,
  就把这个 $\mathcal{D}$ 对应的的区域涂色。而因为 dichotomy 会重叠,多个 $\mathcal{D}$ 的区域也会重叠,而之前 union bound 实际上是
  把这些重叠部分全部都当做了不相交,所以当 $\mathcal{H}$ 变大时,涂色区域会很快占满整张纸。而引入 VC 维之后的上界则能处理这种重叠的情况。
  \item $N$ 变为了 $2N$。因为 $|E_{in}(h) - E_{out}(g)|$ 不仅依赖于 $\mathcal{D}$,也依赖于 $\mathcal{X}$,
  所以引入了一个同样大小的样本 $\mathcal{D}^{\prime}$,变为 $|E_{in}(h) - E_{in}^{\prime}(h)|$。
  \item VC 维很松,因为它不依赖于 $\mathcal{H}$、$\mathcal{P}$ 等,但好处是可以扩展到各种问题和 hypothesis set。
  \item $\sqrt{\frac{8}{N}\ln{\frac{4m_{\mathcal{H}}(2N)}{\delta}}}$ 称为模型复杂度。模型越复杂,$d_{VC}$ 越大,$E_{in}$ 越小,但上界也越松,即 penalty of model complexity。
\end{enumerate}

VC 不等式的证明思路:
\begin{enumerate}
  \item 引入 ghost data。要 bound 的是 $|E_{in} - E_{out}|$,而 $E_{out}$ 未知,所以再抽样一组和 $\mathcal{D}$ 独立同分布且大小相同的 $\mathcal{D}^{\prime}$ 来近似 $E_{out}$,即用 $|E_{in} - E_{in}^{\prime}|$ 来估计之
  \item 对大小为 $2N$ 的数据 $\mathcal{D}$ 和 $\mathcal{D}^{\prime}$ 使用 growth function,限制 hypothesis 数量。
  \item 用 Hoeffding 的另一个定理估计上界。
\end{enumerate}

这东西证明起来还挺复杂,看了几个版本,用 Chebyshev 不等式的版本严重伤害了我们这些低智商码农的感情。要好理解而不失严谨请看 \textit{Learning From Data} 书附录。


\subsection{交集和并集的 VC 维}
交集

$$0 \le d_{VC}(\bigcap_{k=1}^{K}\mathcal{H}_k) \le min\{d_{VC}(\mathcal{H}_k)\}_{k=1}^K$$

并集

$$max\{d_{VC}(\mathcal{H}_k)\}_{k=1}^{K} \le d_{VC}(\bigcup_{k=1}^{K}\mathcal{H}_k) \le K - 1 + \sum_{k=1}^{K}d_{VC}(\mathcal{H}_k)$$

这个要证明一下。假设只有两个 hypothesis set $\mathcal{H}_1$ 和 $\mathcal{H}_2$,并集 $\mathcal{H} = \mathcal{H}_1 \cap \mathcal{H}_2$,VC 维分别是 $d_1$、$d_2$,则显然两者的并集最多能 shatter 两者各自能 shatter 的那么多个点,设为 $m$。即

$$m_{\mathcal{H}}(m) \le m_{\mathcal{H}_1}(m) + m_{\mathcal{H}_2}(m)$$

根据 Sauer's Lemma,

$$m_{\mathcal{H}}(m) \le \sum_{i=0}^{d_1}\binom{m}{i} + \sum_{i=0}^{d_2}\binom{m}{i}$$

替换一下变量,

$$m_{\mathcal{H}}(m) \le \sum_{i=0}^{d_1}\binom{m}{i} + \sum_{i=0}^{d_2}\binom{m}{m-i} = \sum_{i=0}^{d_1}\binom{m}{i} + \sum_{i=m-d_2}^{m}\binom{m}{i}$$

当 $m - d_2 > d_1 + 1$ 即 $m \ge d_1 + d_2 + 2$ 时

$$m_{\mathcal{H}}(m) \le \sum_{i=0}^{d_1}\binom{m}{i} - \binom{m}{d_1+1} = 2^m - \binom{m}{d_1+1} < 2^m$$

也就是说并集的 VC 维应该小于 $m$,而 $m$ 最小为 $d_1 + d_2 + 2$,所以并集 VC 维最大只能取 $d_1 + d_2 + 1$。

参考论文 \href{http://www.davideisenstat.com/cv/EisenstatA07.pdf}{http://www.davideisenstat.com/cv/EisenstatA07.pdf}



\subsection{估算样本数据规模}
根据 VC 不等式,如果要以至少 $1-\delta$ 的 confidence 保证误差不超过 $\epsilon$,则需要

$$\delta \le 4m_{\mathcal{H}(2N)}e^{-\frac{1}{8}N\epsilon^2}$$

变形即得

$$N \ge \frac{8}{\epsilon^2}\ln{\frac{4m_{\mathcal{H}}(2N)}{\delta}}$$

如果用 growth function 的多项式上界 $N^d_{VC} + 1$ 来估计它,则可得

$$N \ge \frac{8}{\epsilon^2}\ln{\frac{4((2N)^{d_{VC}} + 1)}{\delta}}$$

据此便可估计 $N$ 的大小。



\subsection{Noise \& Error}
$\mathcal{D}$ 有噪声时,相当于每个 label $y$ 也是由一个概率分布产生的,即 $\mathbb{P}[y|\mathbf{x}]$。
所以学习算法除了要找 target function $f$,还要决定每个 label 到底取什么值,也就是所谓的 mini-target。
此时 $y$ 不一定告诉你正确的 label,我们看得到的只有被噪声影响的 $f(\mathbf{x})$。
$y$ 不是被 $\mathbf{x}$ 决定($f(\mathbf{x})$),而是受 $\mathbf{x}$ 影响($\mathbb{P}[y|\mathbf{x}]$)。

Error measure 用来衡量我们找到的 hypothesis 所犯的错误多少。常用的 measure 有分类器常用的
0-1 measure $\mathbf{1}\{\tilde{y} \neq y\}$,和回归问题常用的平方误差 $(\tilde{y} - y)^2$。

Error measure 的选择会影响学习算法。例子:三类别分类器,用以上两种 measure 会得到不同的 label 值。

Error measure 也是与算法应用的具体场景相关的。例子:指纹识别,超市根据识别结果判断是否打折,CIA 判断门禁是否通过。
两者需要的 measure 对不同的错判结果有不同的惩罚。设计算法时,可能无法知道这个最佳的 error measure,所以一般会用一个
近似的 $\widehat{err}$。


衡量与 $f$ 的差距仍然要用理想的 $f$。对 0-1 error 而言:
 
\begin{enumerate}
  \item 衡量 $h$ 和 $f$ 的差距要用 $\mathbf{1}\{h(\mathbf{x}) \ne f(\mathbf{x})\}$
  \item $E_{out}(g) = \mathbb{E}_{\mathbf{x}}\left[\mathbf{1}\{f(\mathbf{x}) \neq g(\mathbf{x})\}\right]$
  一定要跟 $f(\mathbf{x})$ 来比较
\end{enumerate}


所谓 weighted classification 就是将模型犯错情况的惩罚考虑进去。例如 CIA 门禁的例子,错放了一个人问题很严重,所以需要加大模型犯这种错误的惩罚。在此情况下,PLA 无需改变,只要数据仍然线性可分即可。对 pocket 算法而言,weighted classification 可以转化为普通的 0-1 error 问题,方法是所谓的 virtual copying,
即,把需要惩罚的错判数据复制很多份,再对这个扩大后的数据集执行修改版 pocket 算法:

\begin{enumerate}
  \item 不用实际复制数据,而是在 pocket 评估 $\mathbf{w}$ 表现时,算入复制后的数据
  \item 随机访问某个点 $\mathbf{x}$ 时,要将复制后的点的访问概率提高 
\end{enumerate}



\section{Week 5}
\subsection{线性回归}
线性回归要求输出某个实数值,而不是 label,可以用线性分类同样的“加权”方法来计算“分数”,从而得到 hypothesis:

$$h(\mathbf{x}) = \mathbf{w}^T\mathbf{x}$$

使用平方误差 $(\hat{y} - y)^2$ 来测算 hypothesis 与实际数据的误差,得到 in- 和 out-of-sample error 的计算方法:

\begin{equation}
\begin{aligned}
E_{in}(\mathbf{w}) &= \frac{1}{N}\sum_{n=1}^{N}(\mathbf{w}^T\mathbf{x}_n - y_n)^2 \\
E_{out}(\mathbf{w}) &= \mathbb{E}_{(\mathbf{x},y)}[(\mathbf{w}^T\mathbf{x} - y)^2]
\end{aligned}
\end{equation}

将 $E_{in}$ 改写为矩阵形式即得

$$E_{in}(\mathbf{w}) = \frac{1}{N}\|\mathbf{Xw} - \mathbf{y}\|^2$$

由于 $E_{in}$ 凸、连续、可微(补充一下证明方法……),所以直接求令其梯度为 $0$ 的 $\mathbf{w}$ 即可:

\begin{equation}
\begin{aligned}
E_{in}(\mathbf{w}) &= \frac{1}{N}\|\mathbf{Xw} - \mathbf{y}\|^2 \\
                   &= \frac{1}{N} \left ( \mathbf{w}^T\mathbf{X}^T\mathbf{Xw} - 2\mathbf{w}^T\mathbf{X}^T\mathbf{y} + \mathbf{y}^T\mathbf{y} \right ) \\
    \nabla E_{in}(\mathbf{w}) &= \frac{2}{N}(\mathbf{X}^T\mathbf{Xw} - \mathbf{X}^T\mathbf{y})
\end{aligned}
\end{equation}

令 $\nabla E_{in}(\mathbf{w}) = 0$ 可得

$$\mathbf{w}_{lin} = (\mathbf{X}^T\mathbf{X})^{-1}\mathbf{X}^T\mathbf{y} = \mathbf{X}^{\dagger}\mathbf{y}$$

这里大部分情况下 $\mathbf{X}^T\mathbf{X}$ 是可逆的,如果不可逆,可以通过其他方式定义 $\mathbf{X}^{\dagger}$,最好用现有的线性代数库来实现。


\subsection{线性回归的可靠性}
线性回归有 closed form,考虑 $E_{in}$ 和 $E_{out}$ 的期望。此处假设每个样本点的噪声都服从 $\epsilon \sim \mathcal{N}(0, \sigma^2)$

首先计算 $\mathbb{E}[E_{in}(\mathbf{w})]$。具体可见 \textit{Learning From Data} 练习 3.3 和 3.4,得到

$E_{in}(\mathbf{w}) = \frac{1}{N}\mathbf{\epsilon}^T(\mathbf{I - H})\mathbf{\epsilon}$。其中 $\mathbf{H}$ 为 hat matrix $\mathbf{X}(\mathbf{X}^T\mathbf{X})^{-1}\mathbf{X}^T$,对称、幂等,用于将 $\mathbf{y}$ 变为 $\hat{\mathbf{y}}$。

故

\begin{equation}
\begin{aligned}
\mathbb{E}[E_{in}(\mathbf{w})] &= \frac{1}{N}\mathbb{E}[\mathbf{\epsilon}^T\mathbf{\epsilon}] - \frac{1}{N}\mathbb{E}[\mathbf{\epsilon}^T\mathbf{H}\mathbf{\epsilon}] \\
&= \sigma^2 - \frac{1}{N}\mathbb{E}[\sum_{i=1}^{N}\epsilon_i^2h_{ii} + \sum_{i \neq j}\epsilon_i\epsilon_jh_{ij}] \\
&= \sigma^2 - \frac{1}{N}(\sum_{i=1}^{N}(\mathbb{E}[\epsilon_i^2]\mathbb{E}[h_{ii}]) + \sum_{i \neq j}\mathbb{E}[\epsilon_i]\mathbb{E}[\epsilon_j]\mathbb{E}[h_{ij}]) \\
&= \sigma^2 - \frac{1}{N}(\sigma^2\mathbb{E}[\sum_{i=1}^{N}h_{ii}] + 0) \\
&= \sigma^2 - \frac{1}{N}(\sigma^2 trace(H)) \\
&= \sigma^2 - \frac{1}{N}(\sigma^2(d + 1)) \\
&= \sigma^2(1 - \frac{d + 1}{N})
\end{aligned}
\end{equation}

同时

$$\mathbb{E}[E_{out}(\mathbf{w})] = \sigma^2(1 + \frac{d + 1}{N})$$

可见当 $N$ 变大时,$E_{in}$ 和 $E_{out}$ 逐渐趋近 $\sigma^2$ 即 noise level,故 generalization error 的期望值为 $\frac{2(d+1)}{N}$。

\subsection{与线性分类的比较}
注意到,平方错误 $(\mathbf{w}^T\mathbf{x} - y)^2$ 是 0-1 错误 $\mathbf{1}\{sign(\mathbf{w}^T\mathbf{x}) \neq y\}$ 的上界,所以在 generalization 时,
线性回归算法得到的 $\mathbf{w}$ 会有更松的上界。如果我们直接对二元分类的问题执行线性回归,
然后返回 $sign(\mathbf{w}_{lin})$ 也是可行的,但相当于用 generalization bound 去换了执行效率。

$\mathbf{w}_{lin}$ 可以作为线性分类的初始 $\mathbf{w}$,从一个较好的 $\mathbf{w}$ 开始迭代,缩短分类算法的运行时间。


\subsection{Logistic 回归 \& 梯度下降法}
分类问题的 mini-target 是 $f(\mathbf{x}) = \mathbb{P}[+1|\mathbf{x}] \in [0, 1]$,但有些问题需要在给定 0-1 label 的基础上直接输出 $\mathbb{P}[+1|\mathbf{x}]$。比如,要预测人患病的概率,我们手里的数据只有某个人是否得病。

所以需要把线性模型中的“加权分值”换算成一个 $[0, 1]$ 上的值,例如 logistic 函数

$$\theta(s) = \frac{1}{1 + e^{-s}}$$

此函数光滑、单调,满足 $\displaystyle \lim_{s \to -\infty} \theta(s) = 0$,$\displaystyle \lim_{s \to +\infty} \theta(s) = 1$,$\theta(0) = \frac{1}{2}$。将其作用到加权分值上即得:

$$h(\mathbf{x}) = \theta(\mathbf{w}^T\mathbf{x})$$

我们要学习的目标为 $f(\mathbf{x}) = \mathbb{P}[+1|\mathbf{x}]$,即在给定数据 $\mathbf{x}$ 的基础上,$+1$(或 $-1$)出现的概率。由于样本中并没有概率,而只有结果,所以相当于样本产生于一个有噪声的目标 $\mathcal{P}(y|\mathbf{x})$:

\begin{equation}
\mathcal{P}(y|\mathbf{x}) =
\begin{cases}
f(\mathbf{x}) & \quad \text{for } y = +1 \\
1 - f(\mathbf{x}) & \quad \text{for } y = -1 \\
\end{cases}
\end{equation}

在此基础上,相当于用 logistic 函数输出的概率 $h(\mathbf{x})$ 来“匹配”训练数据:

\begin{equation}
\mathcal{P}(y|\mathbf{x}) =
\begin{cases}
h(\mathbf{x}) & \quad \text{for } y = +1 \\
1 - h(\mathbf{x}) & \quad \text{for } y = -1 \\
\end{cases}
\end{equation}

注意到 logistic 函数满足 $h(-x) = 1 - h(x)$,所以,综合以上两种情况,即可得 $\mathcal{P}(y|\mathbf{x}) = h(y\mathbf{x}) = \theta(y\mathbf{w}^T\mathbf{x})$。于是可以用似然函数来衡量 hypothesis 与样本的相似程度:

$$\mathcal{L}(\mathbf{w}) = \prod_{n=1}^{N}\mathcal{P}(y_n|\mathbf{x}_n) = \prod_{n=1}^{N}\theta(y_n\mathbf{w}^T\mathbf{x}_n)$$

最大化似然函数等价于最小化:

$$-\frac{1}{N} \ln \left (  \prod_{n=1}^{N} \mathcal{P}(y_n|\mathbf{x}_n) \right ) = \frac{1}{N} \sum_{n=1}^{N} \ln\frac{1}{\mathcal{P}(y_n|\mathbf{x}_n)}$$

最小化这个式子相当于我们把它看做 error measure,即:

$$E_{in}(\mathbf{w}) = \frac{1}{N}\sum_{n=1}^{N}\ln(1 + e^{-y_n\mathbf{w}^T\mathbf{x}_n})$$

这里求和的 error measure $err(\mathbf{w}, \mathbf{x}, y) = \ln(1 + e^{-y\mathbf{w}^T\mathbf{x}})$ 称为 cross entropy error,是一个 pointwise 的 error measure。对它求梯度,可得

$$\nabla E_{in}(\mathbf{w}) = \frac{1}{N}\sum_{n=1}^{N}(-y_n\mathbf{x}_n) \theta(-y_n\mathbf{w}^T\mathbf{x}_n)$$

注意到,这里无法直接令梯度等于 $0$ 来获得一个 analytical solution,但 $E_{in}(\mathbf{w})$ 函数连续、一二阶可微、凸,可用梯度下降法(gradient descent)迭代求解。

梯度下降法类似于 PLA 的迭代过程,每一步对 $\mathbf{w}$ 做适当更新,使 $\mathbf{w}$ 沿着 $E_{in}(\mathbf{w})$ 减小最快(负梯度)方向逐步下降,直至找到局部最小。

假设每次更新方式为

$$\mathbf{w}_{t+1} \leftarrow \mathbf{w}_t + \eta\hat{\mathbf{v}}$$

其中 $\eta > 0$ 为 learning rate,$\hat{\mathbf{v}}$ 为更新方向的单位向量。

更新的目标是让每次更新后 $E_{in}(\mathbf{w})$ 最小。如果 $\eta$ 足够小,在更新的局部,
可以用 Taylor 展开来近似原函数 $E_{in}(\mathbf{w})$,即,在 $\mathbf{w}_t$ 处展开 $E_{in}(\mathbf{w})$:

\begin{equation}
\begin{aligned}
E_{in}(\mathbf{w}) &= E_{in}(\mathbf{w}_t) + \nabla E_{in}(\mathbf{w}_t)^T(\mathbf{w} - \mathbf{w}_t) + O((\mathbf{w} - \mathbf{w}_t)^2)\\
E_{in}(\mathbf{w}_t + \eta\hat{\mathbf{v}}) &= E_{in}(\mathbf{w}_t) + \eta\hat{\mathbf{v}}^T\nabla E_{in}(\mathbf{w}_t) + O(\eta^2) \\
                   &\ge E_{in}(\mathbf{w}_t) - \eta\|\nabla E_{in}(\mathbf{w}_t)\|
\end{aligned}
\end{equation}

最后一步等号成立仅当 $\hat{\mathbf{v}}$ 和 $\nabla E_{in}(\mathbf{w}_t)$ 反向。所以更新方向为

$$\hat{\mathbf{v}} = -\frac{\nabla E_{in}(\mathbf{w}_t)}{\|\nabla E_{in}(\mathbf{w}_t)\|}$$

对于 $\eta$,如果固定取值,则可能因梯度大小不同而引起麻烦,比如函数减小很剧烈时,若 $\eta$ 太小,则可能速度很慢,反之则可能让 $E_{in}(\mathbf{w})$ 不稳定。所以希望根据梯度来动态地决定 $\eta$ 取值。直观上看,让 $\eta \propto \|\nabla E_{in}(\mathbf{w}_t)\|$ 比较好。此时

$$\mathbf{w}_{t+1} \leftarrow \mathbf{w}_t - \eta\frac{\nabla E_{in}(\mathbf{w}_t)}{\|\nabla E_{in}(\mathbf{w}_t)\|}$$ 即相当于

$$\mathbf{w}_{t+1} \leftarrow \mathbf{w}_t - \eta\nabla E_{in}(\mathbf{w}_t)$$

这个新的 $\eta$ 就是原来的 $\displaystyle\frac{\eta}{\|\nabla E_{in}(\mathbf{w}_t)\|}$,称为 fixed learning rate gradient descent。

综合以上步骤即可得到 Logistic 回归的具体操作:

\begin{enumerate}
  \item 初始化 $\mathbf{w}_0$
  \item 计算 $\displaystyle \nabla E_{in}(\mathbf{w}_t) = \frac{1}{N}\sum_{n=1}^{N}(-y_n\mathbf{x}_n) \theta(-y_n\mathbf{w}_t^T\mathbf{x}_n)$
  \item 更新 $\mathbf{w}_{t+1} \leftarrow \mathbf{w}_t - \eta\nabla E_{in}(\mathbf{w}_t)$
  \item 重复迭代直到 $\nabla E_{in}(\mathbf{w}_{t+1}) = 0$ 或足够小
  \item 返回最终的 $\mathbf{w}_{t+1}$ 作为 $g$
\end{enumerate}


从中提取出梯度下降法的一般情况,对于 $F(\mathbf{x})$,用 Taylor 公式在点 $\mathbf{a}$ 处展开得

$$F(\mathbf{x}) = F(\mathbf{a}) + \nabla F(\mathbf{a})(\mathbf{x - a}) + O((\mathbf{x - a})^2)$$

对 $\mathbf{a}$ 附近的 $\mathbf{b} = \mathbf{a} + \eta\mathbf{v}$,其中 $\eta > 1$ 且 $\|\mathbf{v}\| = 1$,有

\begin{equation}
\begin{aligned}
F(\mathbf{b}) &= F(\mathbf{a}) + \eta\mathbf{v}^T\nabla F(\mathbf{a}) + O(\eta^2) \\
              &\le F(\mathbf{a}) - \eta\|\nabla F(\mathbf{a})\|
\end{aligned}
\end{equation}

即,仅当 $\mathbf{v}$ 取负梯度方向时等号成立。所以

$$F(\mathbf{a}) - F(\mathbf{b}) = \eta\|\nabla F(\mathbf{a})\| \ge 0$$

也就是说沿着负梯度方向走,有 $F(\mathbf{a}) \ge F(\mathbf{b})$。


\section{Week 6}
\subsection{线性模型 vs 线性分类}
线性回归和 Logistic 回归都可以用来分类,需要考察一下这样做的好坏。

设 $s = \mathbf{w}^T\mathbf{x}$,则有:
\begin{enumerate}
  \item 线性分类:0-1 错误 $err_{0-1}(s, y) = \mathbf{1}\{sign(ys) \neq 1\}$
  \item 线性回归:平方误差 $err_{SQR}(s, y) = (ys - 1)^2$
  \item Logistic 回归:cross entropy $err_{CE}(s, y) = \ln(1 + e^{-ys})$
\end{enumerate}

将 Logistic 回归的 error measure 换底改写为 scaled cross entropy $err_{SCE}(s, y) = \log_{2}(1 + e^{-ys})$,然后把三个 $ys$ 的函数画图,可以看出:

\begin{enumerate}
  \item 平方误差和 scaled cross entropy 都是 0-1 错误的上界
  \item 平方误差相当于在线性回归中把输出目标定为 $1$,所以抛物线顶点为 $1$,越往两侧 penalty 越大,与 0-1 错误相差也越大
  \item Scaled cross entropy 在 $ys \ll 0$ 时 penalty 很大,与 0-1 错误相差也越大
\end{enumerate}

于是可以得到

\begin{equation}
\begin{aligned}
E_{in}^{0-1}(\mathbf{w}) &\le E_{in}^{SCE}(\mathbf{w}) = \frac{1}{\ln2}E_{in}^{CE}(\mathbf{w}) \\
E_{out}^{0-1}(\mathbf{w}) &\le E_{out}^{SCE}(\mathbf{w}) = \frac{1}{\ln2}E_{out}^{CE}(\mathbf{w})
\end{aligned}
\end{equation}

以上两个 $\le$ 成立是因为线性分类和 Logistic 回归的 $E_{in}$ 都是基于 point-wise error 对所有样本求平均。

有了上面这些关系,套用到 VC 不等式即可知,两个回归模型也可以用于分类,但相应的 generalization bound 更宽松。

相比于线性分类,两个回归模型算法运行更简单,可以用来初始化 $\mathbf{w}_0$。


\subsection{随机梯度下降}
前面说了 Logistic 回归中使用梯度下降法,在每轮迭代中需要计算 $\nabla E_{in}(\mathbf{w})$,以更新 $\mathbf{w}$,而这个计算需要 $O(N)$ 时间遍历所有训练数据,我们希望像 PLA 那样,让每次更新只依赖于一个点,即只用 $O(1)$ 时间。

随机梯度下降法(stochastic gradient descent,SGD)随机选取一个点,只用这个点对原来梯度的贡献来估计原来的梯度,即更新步骤改为

$$\mathbf{w}_{t+1} \leftarrow \mathbf{w}_t + \eta\theta(-y_n\mathbf{w}_t^T\mathbf{x}_n)(y_n\mathbf{x}_n)$$

这样做相当于是把“随机梯度方向”看做“真实梯度方向”加上 zero-mean 的“噪声方向”(补充下理论证明),好处当然是简化了计算,尤其是对训练数据量很大以及 online learning 的情况适用。但停止条件不好找,实践上一般是让算法执行足够长的时间。(估计这个跟 SGD 的随机性有关?)

上面这个更新跟 PLA 很像,把 Logistic 函数换成 $\mathbf{1}\{y_n \neq sign(\mathbf{w}_t^T\mathbf{x}_n)\}$ 即得到 PLA 的更新操作。


\subsection{多类别分类}
\paragraph{One-Versus-All(OVA)} 或称 one-versus-rest。有多个类别时,直接用 hard classifier 来输出 +1/-1 label 可能产生歧义,所以通常是用 soft classifier 输出一个概率,即认为某个点是某个类别的 confidence,然后再汇总。

OVA 对于训练数据中的 $K$ 个类别,用 Logistic 回归训练 $K$ 个分类器,每个分类器使用的训练数据为

$$\mathcal{D}_k = \left \{ (\mathbf{x}_n, y_n^{\prime} = 2 \times \mathbf{1}\{y_n = k\} - 1) \right \}_{n=1}^{N}$$

对于每个数据点,选取这 $K$ 个分类器中 confidence 最大的那个 label:

$$g(\mathbf{x}) = \operatorname*{argmax}_{k \in \mathcal{Y}}\theta(\mathbf{w}_k^T\mathbf{x})$$

上式中,因为 Logistic 函数是单调的,所以可以直接比大小,不用计算一次函数值。

OVA 的好处是简单,而且可以把 Logistic 回归换成任何类似的算法。坏处是当 $K$ 很大时,一般 -1 会比 +1 多很多,所以有可能最后输出的 -1 而非 +1。

\paragraph{One-Versus-One(OVO)} 顾名思义,每次在所有 label 中选出一对来训练二元分类器,故一共要训练 $\displaystyle \binom{N}{2}$ 个分类器。每个分类器使用的训练数据为

$$\mathcal{D}_{k, l} = \{(\mathbf{x}_n, y_n^{\prime}=y_n) \quad | \quad y=k \vee y=l\}_{n=1}^{N}$$

然后用每个分类器的结果对数据点进行投票,票数最高的 label 作为该数据点的 label。

OVO 对于类别数 $K$ 较小的问题比较有效,而且可以搭配任何二元分类算法。缺点是需要花 $O(K^2)$ 空间来保存这些分类器的结果,而且需要的训练量更大。


\subsection{非线性变换}
非线性变换通过变换特征到 $\mathcal{Z}$-space,使非线性模型转换为线性模型。例子:以原点为圆心的圆形 PLA。

一种典型变换:

$$\Phi_2(\mathbf{x}) = (1, x_1, x_2, x_1^2, x_1x_2, x_2^2)$$

包含了所有原始特征的 $0$、$1$、$2$ 次多项式组合。

用非线性变换加线性模型即可实现非线性模型,方法是先变换特征,然后在 $\mathcal{Z}$-space 中用线性模型解决问题,最后把结果对应到原来的数据。

特征变换是常用的手段,尤其是当手上只有 raw feature 时,需要转换而成为具有明确意义的 concrete feature。

非线性变换可能代价高昂。考虑 $\Phi_Q(\mathbf{x})$,将 $d$ 维数据变换为 $Q$ 次多项式:

$$\Phi_Q(\mathbf{x}) = (1, x_1, \dots, x_d, x_1^2, \dots, x_d^Q)$$

变换后数据的维度为 $\displaystyle \binom{Q+d}{Q}$,即 $O(Q^d)$。设变换后维度为 $1 + \tilde{d}$,则相当于训练一个 $\tilde{d}$ 维度的线性模型,其 VC 维也会很大。模型复杂化之后带来的问题是 $E_{in}$ 小了,但 generalize 后 $E_{in}$ 和 $E_{out}$ 差距会较大。

选择模型时要避免“窥探”数据,基于数据特征来选择模型,得到的最终 hypothesis 可能 generalization 表现不好,不要对数据做人肉的推断。

将非线性变换推广到所有的多项式变换,递归地定义如下:

\begin{equation}
\begin{aligned}
\Phi_0(\mathbf{x}) &= (1) \\
\Phi_1(\mathbf{x}) &= (\Phi_0(\mathbf{x}), x_1, \dots, x_d) \\
                   &\dots \\
\Phi_Q(\mathbf{x}) &= (\Phi_{Q-1}(\mathbf{x}), x_1^Q, x_1^{Q-1}x_2, \dots, x_d^Q) \\
\end{aligned}
\end{equation}

它们对应的 hypothesis set 显然满足

$$\mathcal{H}_{\Phi_0} \subset \mathcal{H}_{\Phi_1} \subset \dots \subset \mathcal{H}_{\Phi_Q}$$

所以 VC 维满足

$$d_{VC}(\mathcal{H}_{\Phi_0}) \le  d_{VC}(\mathcal{H}_{\Phi_1}) \le \dots \le d_{VC}(\mathcal{H}_{\Phi_Q})$$

In-sample error 满足

$$E_{in}(g_0) \ge E_{in}(g_1) \ge \dots \ge E_{in}(g_Q)$$

因为 generalization 的问题存在,所以并不是直接用高次的模型就好。一般是从简单有效的模型入手,如果 $E_{in}$ 不理想再逐步提升模型复杂度。



\section{Week 7}
\subsection{Overfitting}
Overfitting 即在降低 $E_{in}$ 的过程中增加了 $E_{out}$。Underfitting 是 $E_{in}$ 本身就比较大,$E_{out}$ 也较高。回忆两个原则:降低 $E_{in}$,保持 $E_{in}$ 和 $E_{out}$ 接近。

可能造成 overfitting 的因素:
\begin{enumerate}
  \item 训练数据少,未展示足够的 pattern
  \item Stochastic noise 太大,干扰了数据本身的 pattern,模型会花力气去模拟噪声
  \item Deterministic noise 太大,即目标太复杂,当前的 $\mathcal{H}$ 竭尽全力也无法模拟 $f$,影响效果相当于噪声
  \item Hypothesis 太强,overkill,比如用 10 次多项式拟合 2 次曲线上的点
\end{enumerate}

针对训练数据和噪声的一些应对措施:
\begin{enumerate}
  \item Data cleaning 修改一些错标的数据
  \item Data pruning 删掉错误数据
  \item Data hinting 基于已有数据生成一些新的数据作为补充,但需要注意新增数据要满足一定的随机性,即产生于 $P(\mathbf{x} | y)$ 否则会影响 generalization。
\end{enumerate}


\subsection{Regularization}
Regularization 即对模型施加一些限制,来减小 overfitting 的可能性。

例子:线性回归 $\mathcal{H}_2$ 可以看做 $\mathcal{H}_{10}$ 的特例,即规定高次权重为 $0$,所以低次多项式的 hypothesis set 是高次的子集,VC 维也更小。这种次数上的限制是一种 hard constraint。通常使用的是 soft constraint,例如限制次数的平方和在一定范围内 $\displaystyle \sum_{q=0}^{10}w_q^2 \le C$,这样形成的 hypothesis set 在取不同的 $C$ 时可能会有交集,但总的来说,$C$ 越大时限制越宽松。此时的 hypothesis set 称为 regularized hypothesis set $\mathcal{H}(C)$

对回归问题而言,现在的优化问题变为,在有限制的条件下最小化 $E_{in}$。写成矩阵形式即

$$\min_{\mathbf{w} \in \mathbb{R}^{\tilde{d}+1}}E_{in}(\mathbf{w}) = \frac{1}{N}(\mathbf{Z}\mathbf{w} - \mathbf{y})^T(\mathbf{Z}\mathbf{w} - \mathbf{y}) \quad \text{s.t.} \quad \mathbf{w}^T\mathbf{w} \le C$$

有约束时,若最优解在约束范围内,则与原问题一致;若最优解在约束范围外,显然最优解应取在
约束区域边缘,否则继续沿负梯度方向往边缘走会得到更优的解。所以只考虑 $\mathbf{w}^T\mathbf{w} \le C$ 即可。根据 Lagrange multiplier,只要沿着约束曲线前进的速度方向
在负梯度方向有分量,就能得到更优的解,所以终止条件是速度方向与负梯度方向垂直,即当前 $\mathbf{w}$
的法线方向与梯度平行,即 $-\nabla E_{in}(\mathbf{w}_{REG}) \propto \mathbf{w}_{REG}$,因为此时约束为一个球,球面任意一点的法线即 $\mathbf{w}$。也就是说,只需要找到 Lagrange multiplier $\lambda > 0$ 和 $\mathbf{w}_{REG}$ 使得

$$\nabla E_{in}(\mathbf{w}_{REG}) + \frac{2\lambda}{N}\mathbf{w}_{REG} = 0$$

两边积分,也就相当于求函数

$$E_{in}(\mathbf{w}_{REG}) + \frac{\lambda}{N}\mathbf{w}^T\mathbf{w}$$

的最小值。这个函数称为 augmented error $E_{aug}$。有约束 $C$ 时的优化问题相当于这个无约束的优化问题。这里 $\lambda > 0$ 是因为我们要限制 $\mathbf{w}$ 且最小化 $E_{in}$,若取 $\lambda < 0$ 则无法得到较小的 $\mathbf{w}$。

如果是线性回归问题,则

$$\nabla E_{in}(\mathbf{w}_{REG}) + \frac{2\lambda}{N}\mathbf{w}_{REG} = 0$$

即

$$\frac{2}{N}(\mathbf{Z}^T\mathbf{Z}\mathbf{w}_{REG} - \mathbf{Z}^T\mathbf{y}) + \frac{2\lambda}{N}\mathbf{w}_{REG} = 0$$

最优解为

$$\mathbf{w}_{REG} = (\mathbf{Z}^T\mathbf{Z} + \lambda\mathbf{I})^{-1}\mathbf{Z}^T\mathbf{y}$$

也称为 ridge regression。Regularizer $\frac{\lambda}{N}\mathbf{w}^T\mathbf{w}$ 称为 weight-decay regularizer,因为它会限制 $\mathbf{w}$ 不要增长太大。

\subsection{与 VC 不等式的联系}
VC 不等式

$$E_{out}(\mathbf{w}) \le E_{in}(\mathbf{w}) + \Omega(\mathcal{H}(C))$$

保证了在有约束 $C$ 时 $E_{out}$ 的上界,这里的 $\Omega$ 项是对整个 hypothesis set 而言,
惩罚较复杂的 hypothesis set。

而使用 augmented error 时,

$$E_{aug}(\mathbf{w}) = E_{in}(\mathbf{w}) + \frac{\lambda}{N}\mathbf{w}^T{\mathbf{w}}$$

直观地讲,如果对某个具体 $h$ 而言,惩罚项 $\frac{\lambda}{N}\Omega(\mathbf{w})$ 是一个对其复杂度加以限制的方式,那么它很有可能对整个 hypothesis set 也能起到限制作用。也就是说,模型的 effective VC dimension 减小了。所以最小化 augmented error 有可能限制 $E_{out}$ 从而达到更好的 generalization。

这里提出 $N$,因为训练数据量越大时,overfitting 的可能性就越小,所以也就越不需要 regularization。

$\lambda$ 和惩罚项的选择取决于 target,但 target 通常是未知的。与 error measure 的选择一样,
可以根据具体问题和优化的需要,选择适当的 $\lambda$ 和惩罚项。



\section{Week 8}
\subsection{模型选择}
在有很多模型可选的情况下,希望从一堆模型里面选出具有最小 $E_{out}$ 的,但显然 $E_{out}$ 无法得知。而且也不能用 $E_{in}$,因为用 $E_{in}$ 的话复杂的模型必然会完胜简单模型,容易导致 overfitting。所以我们需要的是一组与 $\mathcal{D}$ 同分布产生的测试数据。但一般不容易再获取到这样的数据。

所以可以取一种折中的办法,将训练数据分成两部分,一部分做训练,另一部分,称为 validation set 作为测试,来评估各种模型的好坏,称为 validation。

具体做法是从 $\mathcal{D}$ 中随机选择 $K$ 组数据作为 $\mathcal{D}_{val}$。用剩下的数据训练出 $g_m^{-}$,VC 不等式保证了 $E_{out}(g_m^{-})$ 会受 $E_{val}(g_m^{-})$ 的约束。但此时训练数据量变小了,所以 $E_{out}(g_m^{-})$ 应该比实际最好的 $E_{out}(g_m^{*})$ 大。实践上通过 validation 选择出最佳模型后,会再将这个模型作用于整个 $\mathcal{D}$ 再训练一遍,最终输出这最后一个训练结果。

Validation set 的选择要慎重,选多了训练数据量太小,会使得 $E_{out}(g_m^{-})$ 不能很好地近似 $E_{out}$。选少了不能很好地评估模型的好坏。


经验值 $\displaystyle K = \frac{N}{5}$。


\subsection{Leave-One-Out Validation}
做法是取 $K=1$,每次从 $N$ 组数据中选一组作为 validation set,用剩下的 $N-1$ 组数据训练模型,然后做 validation 得到一个 $e_n$。如此重复,对每一组数据都做一遍 validation,再用所有的 $e_n$
取平均,作为对 $E_{out}(g)$ 的估计,此过程称为 cross validation,因为数据会交替地作为训练和验证:

$$E_{loocv}(\mathcal{H}, \mathcal{A}) = \frac{1}{N}\sum_{n=1}^{N}e_n$$

最后取 $E_{loocv}$ 最小的模型即可。

为什么这样做有效?可以取 $E_{loocv}(\mathcal{H, A})$ 的期望:

\begin{equation}
\begin{aligned}
\mathbb{E}_{\mathcal{D}}[E_{loocv(\mathcal(H, A))}] &= \mathbb{E}_{\mathcal{D}}[\frac{1}{N}\sum_{n=1}^{N}e_n] \\
&= \frac{1}{N}\sum_{n=1}^{N}\mathbb{E}_{\mathcal{D}}[e_n] \\
&= \frac{1}{N}\sum_{n=1}^{N}\mathbb{E}_{\mathcal{D}_n}[\mathbb{E}_{(\mathbf{x}_n,y_n)}[err(g_n^{-}(\mathbf{x}_n), y_n)]] \\
&= \frac{1}{N}\sum_{n=1}^{N}\mathbb{E}_{\mathcal{D}_n}[E_{out}(g_n^{-})] \\
&= \frac{1}{N}\sum_{n=1}^{N}\bar{E}_{out}(N - 1) \\
&= \bar{E}_{out}(N - 1)
\end{aligned}
\end{equation}

其中第三步是将每次隔出来的 validation set 与训练数据分开。第四步是相当于在全部 $(\mathbf{x}_n, y_n)$ 上面取误差的期望,即 $E_{out}$。第五步是在全部 $N - 1$ 组数据组成的数据集上取期望,即 $\bar{E}_{out}(N-1)$。于是可得,用 leave-one-out cross validation 来评估模型,可以在期望上接近模型在 $N - 1$ 个数据上的 $E_{out}$。

此方法缺点是计算量太大,每个模型要额外训练很多次。除非是线性回归这种有 analytic solution 的模型,否则不太实用。

\subsection{V-Fold Cross Validation}
前一种方法的改进,目的是减小计算量。方法是将 $\mathcal{D}$ 分成 $V$ 等分,然后再在这 $V$ 等分上做 cross validation,得到

$$E_{CV}(\mathcal{H, A}) = \frac{1}{V}\sum_{v=1}^{V}E_{val}^{(v)}(g_{v}^{-})$$

然后根据这个结果来选择合适的模型。

经验值 $V = 10$。

\subsection{总结}
\begin{enumerate}
  \item V-fold 一般比 single validation 效果好
  \item 通常不需要做 leave-one-out validation
  \item Validation 仍然比实际测试要乐观,也就是说即使 validation 告诉你某个模型很好,实际测试也有可能出乱子。Testing 仍然要以实际测试数据为准。
\end{enumerate}

\subsection{学习准则}
\paragraph{Occam's Razor}
奥卡姆剃刀:如无必要,勿增实体(Entia non sunt multiplicanda praeter necessitatem)

简单即使美。简单的模型有更好的 significance。也就是说,如果这个模型在某个数据集上表现很好,你可以说服自己它确实反映了数据的某些规律。而复杂的模型很可能即使是喂给它完全随机的数据也能表现很好,但实际上数据中并没有什么规律可言。

\paragraph{Sampling Bias}
杜鲁门 vs 杜威 的例子,电话民调选中的大都是有钱人,预测结果当然不准确。

实际上就是得到训练数据和测试数据的分布不一致,得不到 VC 理论的保障。

实践准则是,尽可能地模拟实际的测试场景。例如,如果知道测试场景会倾向于某些数据,则在训练时调整参数,使模型更偏向它们,验证时也用类似于测试场景的数据。

\paragraph{Data Snooping}

为了 VC 理论的安全请不要以任何形式偷看数据,避免 data-driven learning。



\end{document}
